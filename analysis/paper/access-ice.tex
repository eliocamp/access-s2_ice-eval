% Options for packages loaded elsewhere
\PassOptionsToPackage{unicode}{hyperref}
\PassOptionsToPackage{hyphens}{url}
\PassOptionsToPackage{dvipsnames,svgnames,x11names}{xcolor}
%
\documentclass[tc, manuscript]{copernicus}

\usepackage{amsmath,amssymb}
\usepackage{iftex}
\ifPDFTeX
  \usepackage[T1]{fontenc}
  \usepackage[utf8]{inputenc}
  \usepackage{textcomp} % provide euro and other symbols
\else % if luatex or xetex
  \usepackage{unicode-math}
  \defaultfontfeatures{Scale=MatchLowercase}
  \defaultfontfeatures[\rmfamily]{Ligatures=TeX,Scale=1}
\fi
\usepackage{lmodern}
\ifPDFTeX\else  
    % xetex/luatex font selection
\fi
% Use upquote if available, for straight quotes in verbatim environments
\IfFileExists{upquote.sty}{\usepackage{upquote}}{}
\IfFileExists{microtype.sty}{% use microtype if available
  \usepackage[]{microtype}
  \UseMicrotypeSet[protrusion]{basicmath} % disable protrusion for tt fonts
}{}
\makeatletter
\@ifundefined{KOMAClassName}{% if non-KOMA class
  \IfFileExists{parskip.sty}{%
    \usepackage{parskip}
  }{% else
    \setlength{\parindent}{0pt}
    \setlength{\parskip}{6pt plus 2pt minus 1pt}}
}{% if KOMA class
  \KOMAoptions{parskip=half}}
\makeatother
\usepackage{xcolor}
\setlength{\emergencystretch}{3em} % prevent overfull lines
\setcounter{secnumdepth}{5}
% Make \paragraph and \subparagraph free-standing
\makeatletter
\ifx\paragraph\undefined\else
  \let\oldparagraph\paragraph
  \renewcommand{\paragraph}{
    \@ifstar
      \xxxParagraphStar
      \xxxParagraphNoStar
  }
  \newcommand{\xxxParagraphStar}[1]{\oldparagraph*{#1}\mbox{}}
  \newcommand{\xxxParagraphNoStar}[1]{\oldparagraph{#1}\mbox{}}
\fi
\ifx\subparagraph\undefined\else
  \let\oldsubparagraph\subparagraph
  \renewcommand{\subparagraph}{
    \@ifstar
      \xxxSubParagraphStar
      \xxxSubParagraphNoStar
  }
  \newcommand{\xxxSubParagraphStar}[1]{\oldsubparagraph*{#1}\mbox{}}
  \newcommand{\xxxSubParagraphNoStar}[1]{\oldsubparagraph{#1}\mbox{}}
\fi
\makeatother


\providecommand{\tightlist}{%
  \setlength{\itemsep}{0pt}\setlength{\parskip}{0pt}}\usepackage{longtable,booktabs,array}
\usepackage{calc} % for calculating minipage widths
% Correct order of tables after \paragraph or \subparagraph
\usepackage{etoolbox}
\makeatletter
\patchcmd\longtable{\par}{\if@noskipsec\mbox{}\fi\par}{}{}
\makeatother
% Allow footnotes in longtable head/foot
\IfFileExists{footnotehyper.sty}{\usepackage{footnotehyper}}{\usepackage{footnote}}
\makesavenoteenv{longtable}
\usepackage{graphicx}
\makeatletter
\newsavebox\pandoc@box
\newcommand*\pandocbounded[1]{% scales image to fit in text height/width
  \sbox\pandoc@box{#1}%
  \Gscale@div\@tempa{\textheight}{\dimexpr\ht\pandoc@box+\dp\pandoc@box\relax}%
  \Gscale@div\@tempb{\linewidth}{\wd\pandoc@box}%
  \ifdim\@tempb\p@<\@tempa\p@\let\@tempa\@tempb\fi% select the smaller of both
  \ifdim\@tempa\p@<\p@\scalebox{\@tempa}{\usebox\pandoc@box}%
  \else\usebox{\pandoc@box}%
  \fi%
}
% Set default figure placement to htbp
\def\fps@figure{htbp}
\makeatother

\usepackage[inline]{trackchanges}
\makeatletter
\@ifpackageloaded{caption}{}{\usepackage{caption}}
\AtBeginDocument{%
\ifdefined\contentsname
  \renewcommand*\contentsname{Table of contents}
\else
  \newcommand\contentsname{Table of contents}
\fi
\ifdefined\listfigurename
  \renewcommand*\listfigurename{List of Figures}
\else
  \newcommand\listfigurename{List of Figures}
\fi
\ifdefined\listtablename
  \renewcommand*\listtablename{List of Tables}
\else
  \newcommand\listtablename{List of Tables}
\fi
\ifdefined\figurename
  \renewcommand*\figurename{Figure}
\else
  \newcommand\figurename{Figure}
\fi
\ifdefined\tablename
  \renewcommand*\tablename{Table}
\else
  \newcommand\tablename{Table}
\fi
}
\@ifpackageloaded{float}{}{\usepackage{float}}
\floatstyle{ruled}
\@ifundefined{c@chapter}{\newfloat{codelisting}{h}{lop}}{\newfloat{codelisting}{h}{lop}[chapter]}
\floatname{codelisting}{Listing}
\newcommand*\listoflistings{\listof{codelisting}{List of Listings}}
\makeatother
\makeatletter
\makeatother
\makeatletter
\@ifpackageloaded{caption}{}{\usepackage{caption}}
\@ifpackageloaded{subcaption}{}{\usepackage{subcaption}}
\makeatother

\usepackage{bookmark}

\IfFileExists{xurl.sty}{\usepackage{xurl}}{} % add URL line breaks if available
\urlstyle{same} % disable monospaced font for URLs
\hypersetup{
  pdftitle={Evaluation of +S2 sea ice forecast and some other better title},
  colorlinks=true,
  linkcolor={blue},
  filecolor={Maroon},
  citecolor={Blue},
  urlcolor={red},
  pdfcreator={LaTeX via pandoc}}



\title{Evaluation of +S2 sea ice forecast and some other better title}

\Author[1,2]{Elio}{Campitelli}
\Author[1,2]{Ariaan}{Purich}
\Author[1,2]{Julie}{Arblaster}
\Author[3]{Eun-Pa}{Lim}
\Author[3]{Matthew}{Wheeler}
\Author[3]{Phillip}{Reid}


\affil[1]{School of Earth, Atmosphere and Environment, Monash
University, Australia}
\affil[2]{Securing Antarctica's Environmental Future, Monash University,
Australia}
\affil[3]{Bureau of Meteorology}

\runningtitle{+S2 sea ice forecast}

\runningauthor{Campitelli et al.}


\correspondence{Elio\ Campitelli\ (elio.campitelli@monash.edu)}


\begin{document}

\received{}
\pubdiscuss{} %% only important for two-stage journals
\revised{}
\accepted{}
\published{}

%% These dates will be inserted by Copernicus Publications during the typesetting process.


\firstpage{1}
\maketitle
\begin{abstract}
The abstract goes here. It can also be on \emph{multiple lines}.
\end{abstract}
\copyrightstatement{The author's copyright for this publication is
transferred to institution/company.}


\section{Introduction}\label{introduction}

Unlike Arctic sea ice, which has been steadily retreating at least since
the start of satellite records in the early 80s, Antarctic sea ice has
experienced a slightly increasing trend up to 2015, which models
systematically failed to reproduce and puzzled researchers
\citep{hobbs2016}. Then, in 2016 Antarctic sea ice extent dropped
precipitously \citep{turner2017a} and has been at low and record low
levels since \citep{purich2023, hobs2024, gilbert2024, josey2024}, even
more highlighting our lack of understanding of Antarctic sea ice
variability and long-term change \citep{espinosa2024}.

Antarctic sea ice is in a remote location, but the potential impacts of
its variability extends far into the lower latitudes, affecting the
circulation of both the ocean and the atmosphere, ocean carbon uptake
and biological processes. It is likely that a reduction in sea ice
extent would reduce the meridional temperature gradient which, in turn,
would reduce the strength and location of the eddy-driven jet (and the
opposite circulation changes with an increase in sea ice), which is a
crucial control on extratropical weather. In the northern hemisphere
(NH), it has been hypothesised that the reduction in Arctic sea ice has
led to a weaker, wavier jet, increasing the frequency of extreme events
\citep{barnes2015}. There is some evidence of this effect in the
southern hemisphere (SH). Models show a weakened vortex in the middle
atmosphere, increased vertical wave flux and greater zonal wave 1
amplitude in austral spring following a low sea ice maximum. But models
do not fully agree on the exact phase of the zonal wave response, and
observations do not show a clear signal \citep{rea2024}. The jet
response to sea ice loss is also very model dependent, particularly
sensitive to its climatological location, and relatively small compared
with the direct effect of CO2 increase and sea surface temperature (SST)
warming \citep{ayres2019}. On climate scales, it is thought that the
temperature signal of a reduction in Antarctic sea ice mirrors a ``mini
global warming'', and experiments also predict a weakening and
equatorward shift of the SH eddy-driven jet
\citep{raphael2011, england2018, england2020, ayres2022}.

Correctly modelling Antarctic sea is not only necessary for process
understanding and climate projections to inform adaptation strategies
but also for accurate seasonal to sub-seasonal forecasts, which are
crucial for operations in and around the Antarctic continent such as
scientific missions, fisheries and tourism \citep[
\citet{wagner2020}]{desilva2020}. However, producing accurate Antarctic
sea ice forecasts has been challenging due to model biases, inherent
large variability and complexity, and it has lagged behind Arctic sea
ice forecasts \citep{zampieri2019}. Dynamical seasonal forecasts of
summer Antarctic sea ice have been shown to perform worse than
relatively simple statistical methods \citep{massonnet2023}, which also
underscores the need for better understanding of sea ice dynamics.

In this work we evaluate sea ice forecasts produced by the Australian
Community Climate and Earth System Simulator -- seasonal version2
(ACCESS-S2), which is the Australian Bureau of Meteorology (BoM)'s
current operational subseasonal-to-seasonal prediction system, and
compare its forecast skill with that of the former system ACCESS-S1.
This evaluation will inform future work with the prediction system as a
research tool and explore the potential of using its sea-ice forecasts
for decision-making.

\section{Data and methods}\label{data-and-methods}

\subsection{ACCESS-S2}\label{s2}

ACCESS-S2 \citep{wedd2022} became operational in October 2021 by
replacing the ACCESS-S1 system \citep{hudson2017}.. The model components
of both ACCESS-S2 and ACCESS-S1 consist of the Global Atmosphere 6.0
(GA6) \citep{williams2015, waters2017}, Global Land 6.0
\citep{best2011, waters2017}, Global Ocean 5.0
\citep{madec2013, megann2014} and Global Sea Ice 6.0 {[}CICE;
\citet{rae2015}{]}. The atmosphere has a N216 horizontal resolution
(\textasciitilde60km in the mid-latitudes) with 85 vertical levels. The
land model uses the same horizontal grid with 4 soil levels. The ocean
component has a 1/4º resolution with 75 vertical levels. The sea ice
component, based on CICE version 4.1, has the same resolution as the
ocean and 5 sea ice categories as well as an open water category.

Both systems take atmospheric initial conditions derived from
ERA-interim \citep{dee2011} for their hindcasts and from the Bureau's
operational analysis for real-time forecasts. The main difference
between the two are the ocean and sea ice initial conditions. ACCESS-S1
ocean initial conditions come from the Met Office FOAM system, which
uses a multivariate, incremental three-dimensional variational (3D-Var),
first-guess-at-appropriate-time (FGAT) data assimilation scheme
\citep{waters2015} and assimilates sea surface temperature (SST), sea
surface height (SSH), in situ temperature and salinity profiles, and sea
ice concentration. ACCESS-S2, on the other hand, runs from initial
conditions generated by the BoM data assimilation scheme described in
\citet{wedd2022}. This scheme is a weakly coupled ensemble optimal
interpolation method and assimilates temperature and salinity profiles
from EN4 \citep{good2013} for the hindcasts and from the WMO Global
Telecommunication System (GTS) and the Coriolis and USGODAE Global Data
Assembly Centers (GDACs) for the real-time forecasts. SSTs are nudged to
Reynolds OISSTv2.1 \citep{reynolds2007} for the hindcasts and to the
Global Australian Multi-Sensor SST Analysis (GAMSSA; Zhong and Beggs
2008) for the real-time forecasts in areas where SSTs are over 0ºC.
Relevant for this work, sea ice concentrations are not assimilated in
ACCESS-S2.

The ACCESS-S2 hindcast used in this study runs for the period
1981--2018. Each forecast consists of 9 ensemble members built from
three consecutive 3-member forecasts initialised at the first of every
month and the two previous days and run from 279 days. The ACCESS-S1
hindcast is built in the same manner and running for the period
1990--2012.

Anomalies for each hindcast are be taken with respect to a
forecast-dependent climatology for the period 1990--2012. This serves as
a first-order correction of model drift.

\subsection{Verification datasets}\label{verification-datasets}

For verification we use satellite-derived estimates of sea ice
concentration, which estimates the proportion of each grid area that is
covered with ice. Different datasets derived using different algorithms
provide different estimates, each with their own biases and
uncertainties. \citet{meier2019} estimated that the inter-product
uncertainty of sea ice extent is of the order of 1 million \(km^2\). As
will be shown below, this spread is minimal compared with the typical
errors in the ACCESS-S2 and ACCESS-S1 forecasts, so the conclusions are
independent of the dataset used.

We use NOAA/NSIDC's Climate Data Record V4 {[}CDR; \citet{meier2014}{]}
as the primary verification dataset. It takes the maximum value of the
NASA Team \citep{cavalieri1984} and NASA Bootstrap \citep{comiso2023}
sea ice concentration products to reduce their low concentration bias
\citep{meier2014, meier2021}. Both source algorithms use data from the
Scanning Multichannel Microwave Radiometer (SMMR) on the Nimbus-7
satellite and from the Special Sensor Microwave/Imager (SSM/I) sensors
on the Defense Meteorological Satellite Program's (DMSP) -F8, -F11, and
-F13 satellites. The data has a resolution of 25 by 25 km and daily from
1978 onwards.

The European Organisation for the Exploitation of Meteorological
Satellites (EUMETSAT) Ocean and Sea Ice Satellite Application Facility
(OSI SAF) \citep{OSISAF} is another satellite-derived sea ice
concentration product. It is based on mostly the same sensors as the
NOAA CDR but computed independently using different algorithms. Figures
prepared with this dataset are provided in the supplementary material.

\subsection{Error measures}\label{error-measures}

For evaluation purposes, we use a series of measures.

Sea-ice extent is defined as the area of ocean covered with at least
15\% ice. This threshold is motivated by the limitations in satellite
retrieval, which is increasingly unreliable for lower sea ice conditions
(cite -- probably something from NSIDC).

Pan-Antarctic (net) sea ice extent serves as a rough hemispheric measure
of the amount of sea ice, but it does not take into account the spatial
distribution. A model could have relatively accurate extent of the net
ice but with different regional distributions. To account for location
errors, we computed the Root Mean Squared Error (RMSE) of sea ice
concentration anomalies and the Integrated Ice Edge Error {[}IIEE;
\citet{goessling2016}{]}.

We compute RMSE as the square root of the average squared differences
between forecasted and observed sea ice concentration anomalies. We
compute a pan-Antarctic RMSE by averaging over the whole NOAA/NSIDC
CDRV4 southern hemisphere domain, and also a zonally-varying RMSE by
averaging over twenty-four 15° longitude slices around Antarctica.

The IIEE is defined as the area where the model miss-predicts sea ice
concentration being above or below 15\% ice. That is, dichotomise sea
ice concentration into areas with more and less than 15\% sea ice both
in the forecast and observations.

All error measures were computed on the NOAA/NSIDC CDRV4 domain grid and
projection to which model output was bilinearly interpolated.

\section{Results and discussion}\label{results-and-discussion}

\subsection{Bias}\label{bias}

\begin{figure}

\centering{

\pandocbounded{\includegraphics[keepaspectratio]{access-ice_files/figure-pdf/fig-hindcast-extent-1.pdf}}

}

\caption{\label{fig-hindcast-extent}Median sea ice extent for al
hindcasts initialised the first of the month for ACCESS-S2 and ACCESS-S1
in colours representing the start month with the median sea ice extent
of NOAA/NSIDC CDRV4 in the left column and the median difference in the
right column. The NOAA/NSIDC CDRV4 climatology is computed in the period
corresponding to each hindcast. Circles represent the initial conditions
at the first of every month and triangles represent the median value at
the first of every month forecasted with the largest possible lead time.
Colours indicate the initialisation month of the forecast.}

\end{figure}%

Figure~\ref{fig-hindcast-extent} a shows the median sea ice extent (left
column) and median difference with respect of NOAA/NSIDC CDRV4 for the
ACCESS-S2 and ACCESS-S1 hindcasts (right column). Median extent of
initial conditions at the first of every month are indicated with
circles, while median extent at the same date but forecasted with the
largest lead time possible for each model (between 274 and 277 days for
ACCESS-S2 and between 213 and 216 days for ACCESS-S1). At this large
lead time, the the information of the initial conditions is essentially
lost and the forecast reverts to each model's preferred equilibrium
state.

Both models' equilibrium state (triangles) show a negative bias of sea
ice extent, particularly in the late-autumn and winter months. This is
due primarily to faster melt during a longer melt season between January
and March and slower growth during March and April. This is then
partially balanced with faster growth between May and July
(Fig.~\ref{fig-mean-growth}). Many sea ice models exhibit this
systematic underestimation during the sea ice minimum and early freezing
season \citep{massonnet2023} and could indicate problems in the
representation of thermodynamics in the model \citep{zampieri2019}. It
is not surprising that both forecasting systems converge to a similar
climatology because they share the same model formulation.

ACCESS-S2 initial conditions (circles) also have a negative bias,
especially in the late summer-early autumn, while ACCESS-S1 initial
conditions are very close to observations. The faster growth between May
and July reduces the negative bias in ACCESS-S2 for forecasts
initialised from April to June, but it adds a positive bias in
ACCESS-S1.

The difference between the initial conditions and the model preferred
state can be attributed to the effect of data assimilation, which in
ACCESS-S2 is due solely to atmospheric and oceanic data assimilation.
From June to October, in ACCESS-S2 circles are closer to observations
than to the triangles, indicating that the information from the ocean
and atmosphere data assimilation is affecting sea ice and improving the
initial conditions. The rest of the year there is little if any
difference between circles and triangles in ACCESS-S2, indicating that
the ocean and atmosphere data assimilation is not affecting seas ice and
that this component of the model is virtually free-running.

\begin{figure}

\centering{

\pandocbounded{\includegraphics[keepaspectratio]{access-ice_files/figure-pdf/fig-mean-growth-1.pdf}}

}

\caption{\label{fig-mean-growth}Median daily sea ice extent growth of
ACCESS-S1 and ACCESS-S2 hindcasts and observations. Values are smoothed
with a 11-day running mean.}

\end{figure}%

\begin{figure}

\centering{

\pandocbounded{\includegraphics[keepaspectratio]{access-ice_files/figure-pdf/fig-bias-1.pdf}}

}

\caption{\label{fig-bias}Mean sea ice concentrations 1-month lag monthly
mean ACCESS-S2 forecast bias compared with NSIDC.}

\end{figure}%

Figure~\ref{fig-bias} shows the difference of monthly mean sea ice
concentrations between NOAA/NSIDC CDRV4 and ACCESS-S2 hindcasts at the
shortest lag. From October to May, the model underestimates sea ice
concentrations in most regions except for the inner Weddell Sea in April
and May, where sea ice concentrations saturate to 1 both in the
observations and forecasts. In winter, the differences are mostly on the
sea ice edge with light positive bias in the African sector of East
Antarctica and negative bias around the Indian Ocean sector which
partially compensate, resulting in the near-zero extent bias seen in
those months (Fig.~\ref{fig-hindcast-extent}).

\subsection{RMSE}\label{rmse}

\begin{figure}

\centering{

\pandocbounded{\includegraphics[keepaspectratio]{access-ice_files/figure-pdf/fig-extent-anom-1.pdf}}

}

\caption{\label{fig-extent-anom}Monthly mean sea ice extent anomalies
for ACCESS-S1 and ACCESS-S2 (black) and NOAA/NSIDC CDRV4 (blue)
forecasted at selected lead times. The RMSE during the overlapping
period (1990--2013) is shown on the top left.}

\end{figure}%

Figure~\ref{fig-extent-anom} shows monthly sea ice extent anomalies
forecasted at selected lead times. Compared with ACCESS-S1, ACCESS-S2
anomaly forecast is relatively poor even in the first month, which stays
relatively skilful even at lag 3. ACCESS-S2 shows much bigger
variability than observations, with dramatic lows between 1995 and 2007
and highs between 2007 and 2015.

To assess ACCESS-S2 forecasts quantitatively, we compute error measures
for all hindcasts started on the 1st of every month.

\begin{figure}

\centering{

\pandocbounded{\includegraphics[keepaspectratio]{access-ice_files/figure-pdf/fig-rmse-1.pdf}}

}

\caption{\label{fig-rmse}Median RMSE of sea ice concentration anomalies
as a function of forecast lag for all forecast initialised on the first
of each month compared with a reference forecast of persistence of
anomalies.}

\end{figure}%

monthly rmse/corelation at gridpoints per lead time.

Figure~\ref{fig-rmse} shows the median RMSE of sea ice concentration
anomalies for ACCESS-S2 and ACCESS-S1 hindcasts compared with a
benchmark of persistence and climatology. Due to errors in the initial
conditions, it is expected that a persistence forecast would be better
than the model forecast at very short lead times, but that the
persistence forecast errors would grow faster and eventually surpass the
model forecast, after when the model is statistically useful. Here the
persistence errors are almost always lower than the ACCESS-S2 forecast,
indicating that the model does not have skill at any lead time and in
any month. The only exception is around February, where the model has
lower RMSE than the persistence forecast at virtually every lag.

\begin{figure}

\centering{

\pandocbounded{\includegraphics[keepaspectratio]{access-ice_files/figure-pdf/fig-rmse_lon-1.pdf}}

}

\caption{\label{fig-rmse_lon}Median difference between RMSE of ACCESS-S2
forecasts and persistence forecast computed on
\texttt{r\ uniqueN(rmse\_lon\_mean\$lon)} meridional slices
\texttt{r\ diff(unique(rmse\_lon\_mean\$lon)){[}1{]}}º wide.}

\end{figure}%

Even though ACCESS-S2 is no better than persistence at forecasting
pan-Antarctic sea ice concentration anomalies, the quality of the
forecast might vary between regions. To analyse the spatial distribution
of the model error, we computed sea ice concentration anomalies RMSE on
15 meridional slices 24º wide. The median difference between the
forecast and persistence RMSE is shown on Figure~\ref{fig-rmse_lon},
where negative values indicate that the model has lower RMSE than the
benchmark.

The skill shown by ACCESS-S2 at February-March forecasts is evidenced by
a band of negative values across almost all longitudes. The only region
where February-March sea ice is not well forecasted is east of the
Antarctic Peninsula, which is the only region with consistent summer sea
ice. This suggests that the positive skill around the summer minimum
comes from the model correctly forecasting no ice where no ice is ever
found. However, there are other regions of skilful forecasts.

Forecasts initialised on January 1st (Fig.~\ref{fig-rmse_lon} panel a)
show RMSE values below persistence at lags larger than 180 days
(corresponding to July through September) in the Ross and Weddell Sea.
These two regions are also relatively well forecasted from other months.
Notably, May forecasts of Weddell sea ice concentration anomalies
(Fig.~\ref{fig-rmse_lon} panel e) are skilful after about 20 days and
remain so until December. The Weddell Sea is also the region with the
maximum error --around June regardless of initialisation date.

\subsubsection{Comparison with S1}\label{comparison-with-s1}

\begin{figure}

\centering{

\pandocbounded{\includegraphics[keepaspectratio]{access-ice_files/figure-pdf/fig-iiee-1.pdf}}

}

\caption{\label{fig-iiee}Median and 95\% coverage of Integrated Ice Edge
Error as a function of forecast lag for all forecast initialised on the
first of each month for ACCESS-S1 and ACCESS-S2 hindcasts. For each
month, the number in parenthesis indicates the minimum lag at which the
the mean error of each model is not statistically significant at a 1\%
level using a two-sisded t-test.}

\end{figure}%

To compare ACCESS-S2 with ACCESS-S1, we computed the IIEE for both
models. This error measure is shown in Figure~\ref{fig-iiee} for
forecasts initialised at the first of every month at all lead times.
ACCESS-S1 has lower error at lead times up to 60 days at all months,
with the errors converging to ACCESS-S2 as the forecast goes on. The
time to convergence depends on the month and it can be as short as two
weeks June and July to as long as 160 days for forecasts initialised in
February. Since the only difference between these forecasts are the
initial conditions, this timescale is an indication of the memory of sea
ice to initial conditions --at least from October to March when the data
assimilated form the other components has little to no influence on sea
ice.

Also evident in Figure~\ref{fig-iiee} is the difference in the error
spread at short lags between ACCESS-S2 and ACCESS-S1. In all months
ACCESS-S1 has much narrower error spread at the first day of the
forecasts. The error in the initial conditions is not only small, but
also fairly constant. This spread then grows towards a climatological
spread as errors accumulate differently in different forecasts. For
ACCESS-S2, this is true only between July and October, approximately.
For all other months, the error spread is more or less stable throughout
the forecast window, indicating that not only the initial error is high,
but it is not constant.

\begin{figure}

\centering{

\pandocbounded{\includegraphics[keepaspectratio]{access-ice_files/figure-pdf/fig-iiee-variance-1.pdf}}

}

\caption{\label{fig-iiee-variance}Mean spread of IIEE at different lags
for different models.}

\end{figure}%

The large initial error spread could be due either to large spread of
ensemble members or to large spread of individual forecasts.
Figure~\ref{fig-iiee-variance} splits the IIEE variance for each lead
time into the mean variance of each individual forecast and the variance
of the mean error of each individual forecast, which adds up to the
total variance. The average variance of each forecast is almost
identical between the two forecast systems in all months. This shows
that the ensemble spread of individual forecasts evolves identically,
which, again, is not unexpected because both systems share the same
model formulation. This also shows that the perturbation scheme in
ACCESS-S2 is comparable to the one in ACCESS-S1.

On the other hand, the spread of the mean error is always larger in
ACCESS-S2 than ACCESS-S1. The difference is particularly large at short
lead times in some months, which coincide with the ones in which the
data assimilation scheme is not influencing sea ice initial conditions.

\subsection{Conclusions}\label{conclusions}

Sea ice forecasts from the ACCESS-S2 model show a significant low extent
bias, particularly during late summer and early autumn. This bias is
attributed to a faster and longer melt season between January and March,
and slower growth between March and April. This underestimation during
the minimum and early freezing season is a common issue in many
seasonal-to-subseasonal (S2S) systems, suggesting potential problems
either with the model's thermodynamic representation or with short wave
radiation forcing, as shown in other climate models. Even though
ACCESS-S2 shares the same model formulation with ACCESS-S1, the latter
does not suffer from this bias. This is due to the assimilation of sea
ice concentrations into the initial conditions, which successful
corrects for the negative bias in the model. Our analysis suggests that
the data assimilation system in ACCESS-S2 is only effectively
influencing sea ice initial conditions from June to October, while the
rest of the year, the sea ice component runs virtually free, reverting
to its biased equilibrium state.

Analysis of the error spread shows that ACCESS-S2 initial conditions
from December to May not only have large errors, but that the initial
error spread is very large compared with ACCESS-S1. This spread is not
due to the perturbation scheme, since the mean error variance for
individual forecasts is low and comparable with ACCESS-S1. Instead, it
is due to large variance of the mean error of individual forecasts,
which is comparable to the climatology spread. This is further evidence
that individual initial conditions are not being affected by the data
assimilation scheme.

Based on the observation that ACCESS-S2 sea ice initial conditions are
essentially not initialised, comparing its forecasts with ACCESS-S1's
allows us to estimate the time-scale for which initial conditions are
important. This highlights February initial conditions as a crucial for
determining sea ice evolution at least up to late June. Arctic sea ice
forecast also show greater sensitivity to initial conditions in boreal
summer compared with boreal winter \citep{day2014, bunzel2016}, so
similar mechanism might be playing a role.

Although forecasts are not skilful for forecasting pan-Antarctic sea ice
concentrations, there are some areas where the model does show skill.
The Weddell Sea, and the Ross Sea to a lesser extent are particularly
well forecasted between June and November for forecasts initialised from
January to June. These regions have been recognised as a regions of high
predictability thanks to persistent and eastwardly advected upper ocean
heat content anomalies \citep{bushuk2021}. Sea surface temperatures are
assimilated by ACCESS-S2 only in areas with temperatures greater than
0°C and errors don't show a clear easterly-propagating signal, so it is
not clear if ACCESS-S2 is leveraging the same source of predictability..

\begin{center}\rule{0.5\linewidth}{0.5pt}\end{center}

\section{References}\label{references}

\renewcommand{\bibsection}{}
\bibliography{references.bib}

\section{Suplementary figures}\label{suplementary-figures}



\codedataavailability{use this to add a statement when having data sets
and software code
available} %% use this section when having data sets and software code available

\sampleavailability{use this section when having geoscientific samples
available} %% use this section when having geoscientific samples available

\videosupplement{use this section when having video supplements
available} %% use this section when having geoscientific samples available

%%%%%%%%%%%%%%%%%%%%%%%%%%%%%%%%%%%%%%%%%%
%% optional

%%%%%%%%%%%%%%%%%%%%%%%%%%%%%%%%%%%%%%%%%%
\appendix
\section{Figures and tables in appendices}

Regarding figures and tables in appendices, the following two options
are possible depending on your general handling of figures and tables in
the manuscript environment:

\subsection{Option 1}

If you sorted all figures and tables into the sections of the text,
please also sort the appendix figures and appendix tables into the
respective appendix sections. They will be correctly named
automatically.

\subsection{Option 2}

If you put all figures after the reference list, please insert appendix
tables and figures after the normal tables and figures.

To rename them correctly to A1, A2, etc., please add the following
commands in front of them: \texttt{\textbackslash{}appendixfigures}
needs to be added in front of appendix figures
\texttt{\textbackslash{}appendixtables} needs to be added in front of
appendix tables

Please add \texttt{\textbackslash{}clearpage} between each table and/or
figure.
\noappendix

%%%%%%%%%%%%%%%%%%%%%%%%%%%%%%%%%%%%%%%%%%
\authorcontribution{Contributions} %% optional section

%%%%%%%%%%%%%%%%%%%%%%%%%%%%%%%%%%%%%%%%%%
\competinginterests{The authors declare no competing
interests.} %% this section is mandatory even if you declare that no competing interests are present

%%%%%%%%%%%%%%%%%%%%%%%%%%%%%%%%%%%%%%%%%%
\disclaimer{We like Copernicus.} %% optional section

%%%%%%%%%%%%%%%%%%%%%%%%%%%%%%%%%%%%%%%%%%
\begin{acknowledgements}
Acknowledgements will gohere
\end{acknowledgements}

\bibliographystyle{copernicus}

\end{document}

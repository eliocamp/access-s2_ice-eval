% Options for packages loaded elsewhere
% Options for packages loaded elsewhere
\PassOptionsToPackage{unicode}{hyperref}
\PassOptionsToPackage{hyphens}{url}
\PassOptionsToPackage{dvipsnames,svgnames,x11names}{xcolor}
%
\documentclass[bg, manuscript]{copernicus}
\usepackage{xcolor}
\usepackage{amsmath,amssymb}
\setcounter{secnumdepth}{5}
\usepackage{iftex}
\ifPDFTeX
  \usepackage[T1]{fontenc}
  \usepackage[utf8]{inputenc}
  \usepackage{textcomp} % provide euro and other symbols
\else % if luatex or xetex
  \usepackage{unicode-math} % this also loads fontspec
  \defaultfontfeatures{Scale=MatchLowercase}
  \defaultfontfeatures[\rmfamily]{Ligatures=TeX,Scale=1}
\fi
\usepackage{lmodern}
\ifPDFTeX\else
  % xetex/luatex font selection
\fi
% Use upquote if available, for straight quotes in verbatim environments
\IfFileExists{upquote.sty}{\usepackage{upquote}}{}
\IfFileExists{microtype.sty}{% use microtype if available
  \usepackage[]{microtype}
  \UseMicrotypeSet[protrusion]{basicmath} % disable protrusion for tt fonts
}{}
\makeatletter
\@ifundefined{KOMAClassName}{% if non-KOMA class
  \IfFileExists{parskip.sty}{%
    \usepackage{parskip}
  }{% else
    \setlength{\parindent}{0pt}
    \setlength{\parskip}{6pt plus 2pt minus 1pt}}
}{% if KOMA class
  \KOMAoptions{parskip=half}}
\makeatother
% Make \paragraph and \subparagraph free-standing
\makeatletter
\ifx\paragraph\undefined\else
  \let\oldparagraph\paragraph
  \renewcommand{\paragraph}{
    \@ifstar
      \xxxParagraphStar
      \xxxParagraphNoStar
  }
  \newcommand{\xxxParagraphStar}[1]{\oldparagraph*{#1}\mbox{}}
  \newcommand{\xxxParagraphNoStar}[1]{\oldparagraph{#1}\mbox{}}
\fi
\ifx\subparagraph\undefined\else
  \let\oldsubparagraph\subparagraph
  \renewcommand{\subparagraph}{
    \@ifstar
      \xxxSubParagraphStar
      \xxxSubParagraphNoStar
  }
  \newcommand{\xxxSubParagraphStar}[1]{\oldsubparagraph*{#1}\mbox{}}
  \newcommand{\xxxSubParagraphNoStar}[1]{\oldsubparagraph{#1}\mbox{}}
\fi
\makeatother


\usepackage{longtable,booktabs,array}
\usepackage{calc} % for calculating minipage widths
% Correct order of tables after \paragraph or \subparagraph
\usepackage{etoolbox}
\makeatletter
\patchcmd\longtable{\par}{\if@noskipsec\mbox{}\fi\par}{}{}
\makeatother
% Allow footnotes in longtable head/foot
\IfFileExists{footnotehyper.sty}{\usepackage{footnotehyper}}{\usepackage{footnote}}
\makesavenoteenv{longtable}
\usepackage{graphicx}
\makeatletter
\newsavebox\pandoc@box
\newcommand*\pandocbounded[1]{% scales image to fit in text height/width
  \sbox\pandoc@box{#1}%
  \Gscale@div\@tempa{\textheight}{\dimexpr\ht\pandoc@box+\dp\pandoc@box\relax}%
  \Gscale@div\@tempb{\linewidth}{\wd\pandoc@box}%
  \ifdim\@tempb\p@<\@tempa\p@\let\@tempa\@tempb\fi% select the smaller of both
  \ifdim\@tempa\p@<\p@\scalebox{\@tempa}{\usebox\pandoc@box}%
  \else\usebox{\pandoc@box}%
  \fi%
}
% Set default figure placement to htbp
\def\fps@figure{htbp}
\makeatother





\setlength{\emergencystretch}{3em} % prevent overfull lines

\providecommand{\tightlist}{%
  \setlength{\itemsep}{0pt}\setlength{\parskip}{0pt}}





\usepackage[inline]{trackchanges}
\usepackage[section]{placeins}
\makeatletter
\@ifpackageloaded{caption}{}{\usepackage{caption}}
\AtBeginDocument{%
\ifdefined\contentsname
  \renewcommand*\contentsname{Table of contents}
\else
  \newcommand\contentsname{Table of contents}
\fi
\ifdefined\listfigurename
  \renewcommand*\listfigurename{List of Figures}
\else
  \newcommand\listfigurename{List of Figures}
\fi
\ifdefined\listtablename
  \renewcommand*\listtablename{List of Tables}
\else
  \newcommand\listtablename{List of Tables}
\fi
\ifdefined\figurename
  \renewcommand*\figurename{Figure}
\else
  \newcommand\figurename{Figure}
\fi
\ifdefined\tablename
  \renewcommand*\tablename{Table}
\else
  \newcommand\tablename{Table}
\fi
}
\@ifpackageloaded{float}{}{\usepackage{float}}
\floatstyle{ruled}
\@ifundefined{c@chapter}{\newfloat{codelisting}{h}{lop}}{\newfloat{codelisting}{h}{lop}[chapter]}
\floatname{codelisting}{Listing}
\newcommand*\listoflistings{\listof{codelisting}{List of Listings}}
\makeatother
\makeatletter
\makeatother
\makeatletter
\@ifpackageloaded{caption}{}{\usepackage{caption}}
\@ifpackageloaded{subcaption}{}{\usepackage{subcaption}}
\makeatother
\usepackage{bookmark}
\IfFileExists{xurl.sty}{\usepackage{xurl}}{} % add URL line breaks if available
\urlstyle{same}
\hypersetup{
  pdftitle={The Importance of Initial Conditions in Seasonal Predictions of Antarctic Sea Ice},
  pdfkeywords={sea ice, seasonal predictability, initial
conditions, forecasting},
  colorlinks=true,
  linkcolor={blue},
  filecolor={Maroon},
  citecolor={Blue},
  urlcolor={red},
  pdfcreator={LaTeX via pandoc}}



\title{The Importance of Initial Conditions in Seasonal Predictions of
Antarctic Sea Ice}

\Author[1, 2]{Elio}{Campitelli}
\Author[1, 2]{Ariaan}{Purich}
\Author[1, 2]{Julie}{Arblaster}
\Author[3]{Eun-Pa}{Lim}
\Author[3]{Matthew C.}{Wheeler}
\Author[3]{Phillip}{Reid}


\affil[1]{School of Earth, Atmosphere and Environment, Monash
University, Kulin Nations, Clayton, Victoria, Australia.}
\affil[2]{ARC Special Research Initiative for Securing Antarctica's
Environmental Future, Clayton, Kulin Nations, Victoria, Australia}
\affil[3]{Research, Bureau of Meteorology, Melbourne, Australia.}

\runningtitle{The Importance of Initial Conditions in Seasonal
Predictions of Antarctic Sea Ice}

\runningauthor{Campitelli et al.}


\correspondence{Elio\ Campitelli\ (elio.campitelli@monash.edu)}


\begin{document}

\received{}
\pubdiscuss{} %% only important for two-stage journals
\revised{}
\accepted{}
\published{}

%% These dates will be inserted by Copernicus Publications during the typesetting process.


\firstpage{1}
\maketitle
\begin{abstract}
Accurate Antarctic sea-ice forecasts are crucial for climate monitoring
and operational planning, yet they remain challenging due to model
biases and complex ice-ocean-atmosphere interactions. The two versions
of the Australian Bureau of Meteorology's ACCESS seasonal forecast
system, ACCESS-S1 and ACCESS-S2, use identical model configuration and
differ only in their initial conditions; primarily in that ACCESS-S2
does not assimilate sea-ice observations, whereas ACCESS-S1 does.\\
This provides a convenient opportunistic experiment to assess the role
of initial conditions on Antarctic sea-ice forecasts using more than 20
years of fully coupled simulations with two 9-member ensembles. Our
analysis reveals that both systems experience an extended melt season
and delayed growth phase compared with observations. This leads to a
significant negative sea-ice extent bias, which is corrected only in
ACCESS-S1 by the data assimilation system. The impact of the differing
initial conditions on forecast errors varies dramatically by season:
summer and autumn initial conditions (January-April) provide predictive
skill for up to three months, with February initial conditions being
particularly crucial. In contrast, winter forecasts of the two systems
are statistically indistinguishable after only two weeks. Regional
analysis of forecast skill suggests that this winter predictability
barrier is most dramatic over East Antarctica, where even ACCESS-S1
shows negative skill. These findings highlight the critical importance
of comprehensive year-round sampling in predictability studies and
suggest that operational sea-ice data assimilation efforts should
prioritise the summer-autumn period when initial conditions have maximum
impact on forecast skill.
\end{abstract}


\section{Introduction}\label{introduction}

Accurately modelling Antarctic sea ice is essential for understanding
processes and improving climate projections to inform adaptation
strategies. Accurate seasonal to sub-seasonal forecasts are also crucial
for operation contingency planning in and around the Antarctic
continent, including scientific missions, fisheries, and tourism
\citep{desilva2020, wagner2020}. Improvements in modelled sea-ice might
also help improve weather forecasts over and away from sea-ice regions
\citep{rinke2006, wang2024, semmler2016}.

However, progress in Antarctic sea-ice forecasting system has lagged
behind Arctic sea-ice forecasts due to model biases, and inherent large
variability and complexity \citep{zampieri2019, gao2024a}. Dynamical
seasonal forecasts of summer Antarctic sea ice have been shown to
perform worse than relatively simpler statistical methods
\citep{massonnet2023} and machine learning approaches (e.g.
\citet{dong2024}, \citet{lin2025}), which also underscores the need for
better understanding and physical modelling of sea-ice dynamics, and
drivers of its variability.

Good initial conditions are generally required for a good forecast,
however, it is not entirely known to what extent accurate sea-ice
initial conditions affect the quality of the forecast and at what
timescales. Exploring seasonal predictions of Arctic sea ice,
\citet{guemas2016} found that sea-ice initial conditions are important
in autumn to predict summer sea ice, but the impact wasn't as dramatic
when predicting winter sea ice. \citet{day2014} also found
seasonally-varying differences in the effect of initialisation, noting
that accurate Arctic sea-ice thickness leads to improved sea-ice
forecasts initialised in July but not when initialised in January.

For the Antarctic, \citet{holland2013} studied the initial-value
predictability of Antarctic sea ice in a perfect model study using the
CCSM3 model. They found that sea-ice and ocean initial conditions
provide predictive information to forecast sea-ice edge location several
months in advance and that some predictability is retained for up to two
years thanks to ocean heat content anomalies that are advected eastward.
This is in contrast with \citet{marchi2020}, who ran perfect model
experiments to argue that uncertainty in the predicted atmospheric state
and evolution is the main driver of uncertainty in Antarctic sea-ice
extent prediction on seasonal timescales, with sea-ice and ocean initial
conditions having lesser importance. More recently, \citet{morioka2022}
studied decadal forecasts of Antarctic sea ice and found that
initialising ocean and sea ice improved the correlation between
simulated and observed sea-ice concentration evolution in the
Amundsen--Bellingshausen Sea. It is hard to compare these studies since
they are based on forecasts initialised at different times of the year
and different frameworks: \citet{holland2013} ran 20 ensemble members
initialised on the 1st of January of a particular year,
\citet{marchi2020} ran forecasts from the 1st of March and 1st of
September, and \citet{morioka2022} ran forecasts only from the 1st of
March. \citet{marchi2020} also used a coupled ocean--sea-ice model
instead of a fully coupled model like \citet{holland2013} did.
\citet{morioka2022} used observed sea-ice initial conditions and
compared with observations, while \citet{marchi2020} and
\citet{holland2013} were perfect model studies.

In October 2021 the Australian Bureau of Meteorology (BoM) upgraded the
Australian Community Climate and Earth System Simulator -- Seasonal
(ACCESS-S) from version S1 to S2. While the base model remained the
same, the change in version was focused on using ocean, sea-ice and land
initial conditions generated by the BoM instead of depending on the UK
Met Office. Crucially, compared to ACCESS-S1, ACCESS-S2 does not
assimilate sea-ice observations, so sea ice is only affected by the
ocean and atmospheric data assimilation via the coupled integration.

Since model configuration is identical between ACCESS-S1 and ACCESS-S2,
they form a sort of ``opportunistic experiment'' where the same
forecasting model was run over a long period of time with multiple
ensemble forecasts initialised throughout the year, with the only
difference being the initial conditions. This provides an opportunity to
test the effect of sea-ice initial conditions on the forecast of sea-ice
concentrations and the climate.

In this study we compare sea-ice hindcasts produced by ACCESS-S1 and
ACCESS-S2. We focus on seasonality of errors and biases and the effect
of the data assimilation system. This comparison will inform future work
with the prediction system as a research tool to better understand the
dynamics and variability of the Antarctic sea ice and its impacts on the
climate system as well as to explore the potential of using its sea-ice
forecasts for decision-making. The work will also serve as a benchmark
for future prediction systems to attempt to improve upon.

\section{Data and methods}\label{data-and-methods}

\subsection{Forecasting systems}\label{forecasting-systems}

ACCESS-S2 \citep{wedd2022} is the Bureau of Meteorology's seasonal
forecast system which became operational in October 2021, replacing the
ACCESS-S1 system \citep{hudson2017}. The model components of both
ACCESS-S2 and ACCESS-S1 are identical with the same numbers of levels
and resolution. They consist of the Global Atmosphere 6.0 (GA6)
\citep{williams2015, waters2017}, the Unified Model's Global Land 6.0
\citep{best2011, waters2017}, NEMO Global Ocean 5.0
\citep{gurvan2013, megann2014} and Global Sea Ice 6.0 {[}CICE;
\citet{rae2015}{]}. The atmosphere has a N216 horizontal resolution
(\textasciitilde60 km in the mid-latitudes) with 85 vertical levels. The
land model uses the same horizontal grid as the atmosphere with four
soil levels. The ocean component has a nominal horizontal resolution of
1/4° with 75 vertical levels. The sea-ice component, based on CICE
version 4.1, has the same resolution as the ocean component and five
sea-ice thickness categories as well as an open water category.

Both systems take atmospheric initial conditions derived from
ERA-interim \citep{dee2011} for their hindcasts. The main difference
between the hindcasts of the two systems are the ocean and sea-ice
initial conditions.

ACCESS-S1's ocean and sea-ice initial conditions come from the Met
Office FOAM system, which uses a multivariate, incremental
three-dimensional variational (3D-Var), first-guess-at-appropriate-time
(FGAT) data assimilation scheme \citep{waters2015} and assimilates sea
surface temperature (SST), sea surface height (SSH), in situ temperature
and salinity profiles, and satellite observations of sea-ice
concentration using the EUMETSAT OSISAF product described in the next
section. The way sea-ice thickness is handled is that sea-ice
concentration innovations are either added to the first ice category
with a fixed thickness of 50cm or removed from the thinnest category
first and from thicker categories if needed.

ACCESS-S2, on the other hand, is initialised from ocean conditions
generated by the BoM weakly coupled ensemble data assimilation scheme
described in \citet{wedd2022}. This scheme uses an optimal interpolation
method and assimilates temperature and salinity profiles from EN4
\citep{good2013}. SSTs are nudged to Reynolds OISSTv2.1
\citep{reynolds2007} in areas where SSTs are over 0°C and Sea Surface
Salinity is weakly nudged to the World Ocean Atlas 2013 climatology
\citep{zweng2013}.

Of most relevance for this work, sea-ice concentrations are not
assimilated in ACCESS-S2. Assimilation cycles are performed daily. The
coupled model runs for 24 hours initialised from the previous cycle.
Then the restart file fields of the ocean component are used as first
guess in the data assimilation cycle and the innovations are used to
build the next ocean initial conditions for the following cycle. The
atmosphere fields from that daily integration are not used and instead
the model atmosphere is initialised using ERA-Interim. The sea-ice
initial conditions for the next cycle are the unaltered output of the
previous daily integration. Then the cycle starts again and the coupled
model runs for another 24 hours. During this integration the sea-ice
component is affected by the ocean innovations and the new atmosphere
initial conditions via the coupler.

The ACCESS-S1 hindcast set is made up of nine members created by
perturbing the atmospheric fields only with a random field perturbation
\citep{hudson2017} and runs for 217 days for the period 1990--2012
initialised at the first of every month. The ACCESS-S2 hindcast set used
in this study runs for the period 1981--2018. Ensemble members are
created in the same manner as ACCESS-S1 members, however, due to
computing cost limitations, only three members per forecast
initialisation date were run for 279 days. Bigger ensembles were
generated by aggregating several three-member ensembles initialised on
successive days \citep{wedd2022}. Here, we build a nine-member
time-lagged ensemble from three consecutive three-member forecasts
initialised at the first of every month and the two previous days and
run for 279 days. We analyse the ensemble mean hindcasts unless
otherwise specified.

Anomalies for each hindcast set are taken with respect to their own
climatology specific to each initialisation date and forecast lead time,
for the period 1990--2012. This serves as a first-order correction of
model bias and drift. For monthly means, we define ``0 lead time
months'' as the monthly mean forecast of the same month of
initialisation.

Besides sea-ice concentration, we also analyse mean sea-ice thickness,
which we compute as total sea-ice volume divided by total sea-ice area.

\subsection{Verification datasets}\label{verification-datasets}

For verification we use satellite-derived sea-ice concentration, which
estimates the proportion of each grid area that is covered with ice.
Datasets derived using different algorithms and satellite platforms,
each have their own biases and uncertainties. Estimates of inter-product
uncertainty of sea-ice extent (SIE, defined here as the total region of
the Southern Ocean with at least 15\% sea-ice cover) are of the order of
0.5 million \(km^2\) \citep{meier2019}. As will be shown below, this
spread is minimal compared with the typical errors in the ACCESS-S2 and
ACCESS-S1 forecasts, so the overall conclusions of this study are
independent of the verification dataset used.

We use NOAA/NSIDC's Climate Data Record V4 {[}CDR; \citet{meier2014}{]}
as the primary sea-ice verification dataset. It takes the maximum value
of the NASA Team \citep{cavalieri1984} and NASA Bootstrap
\citep{comiso2023} sea-ice concentration products to reduce their low
concentration bias \citep{meier2014, meier2021}. Both source algorithms
use data from the Scanning Multichannel Microwave Radiometer (SMMR) on
the Nimbus-7 satellite and from the Special Sensor Microwave/Imager
(SSM/I) sensors on the Defense Meteorological Satellite Program's (DMSP)
-F8, -F11, and -F13 satellites. The data have a spatial resolution of 25
by 25 km and daily from November 1978 onwards.

The European Organisation for the Exploitation of Meteorological
Satellites (EUMETSAT) Ocean and Sea Ice Satellite Application Facility
{[}OSI; \citet{OSISAF}{]} based on the SSMIS sensor is another
satellite-derived sea-ice concentration product. It is based on mostly
the same sensors as the NOAA CDR but computed independently using
different algorithms. Figures prepared with this dataset are provided in
the appendix and do not differ significantly from the ones prepared
using CDR.

\subsection{Error measures}\label{error-measures}

For evaluation purposes, we use a series of measures. Sea-ice extent is
defined as the area of the ocean at least 15\% covered by sea-ice. This
threshold is motivated by the limitations in satellite retrieval, which
is increasingly unreliable for lower sea-ice concentrations
\citep{cavalieri1991}. We evaluate sea-ice extent anomaly hindcasts by
their Root Mean Squared Error (RMSE) and correlation coefficient. The
hindcast ensemble mean sea-ice extent is defined as the sea-ice extent
of the ensemble mean hindcast.

Pan-Antarctic (net) sea-ice extent serves as a hemispheric measure of
the amount of sea ice, but it does not take into account the spatial
distribution. A model could have a relatively accurate extent of the net
ice but with different regional distributions. To account for location
errors, we computed the RMSE of grid-point sea-ice concentration
anomalies.

We compute RMSE as the square root of the area-averaged squared
differences between grid-point forecasted and observed sea-ice
concentration anomalies. We compute a pan-Antarctic RMSE by averaging
over the whole NOAA/NSIDC CDRV4 Southern Hemisphere domain, and also a
zonally-varying RMSE computed over 15 longitude slices 24° wide around
Antarctica.

All error measures were computed on the NOAA/NSIDC CDRV4 domain grid, to
which model output was bilinearly interpolated. Note that the ACCESS
CICE model grid has resolution between two and three times higher than
NOAA/NSIDC CDRV4.

Forecast errors are also compared with hypothetical forecasts based on
the persistence of anomalies and on climatology. The persistence
forecast is generated by extending the observed sea-ice concentration
anomalies the day of the forecast initialisation and comparing it with
the actual anomalies observed. The climatological forecast error is
computed as the standard deviation of daily anomalies.

As a measure of forecast improvement over the hypothetical forecast, we
use the skill score \citep{murphy1985}, defined as

\[
S = 1 - \frac{RMSE_{f}}{RMSE_{r}}
\]

Where \(RMSE_{f}\) is the RMSE of the forecast, \(RMSE_{r}\) is the RMSE
of the reference forecast. Negative skill score indicates that the
forecast is worse than the reference forecast while positive values
indicate an improvement. A perfect forecast would have zero RMSE and
thus a skill score of 1.

Furthermore, we also compute sea-ice concentration bias, defined as the
mean difference between the forecast and observations.

\subsection{Computational procedures}\label{computational-procedures}

We performed all analyses in this paper using the R programming language
\citep{rcoreteam2020}, using data.table \citep{dowle2020} and metR
\citep{campitelli2020b} packages. Significant processing was performed
using the CDO command line operators \citep{schulzweida2023}. All
graphics are made using ggplot2 \citep{wickham2009}. The paper was
rendered using knitr and Quarto \citep{xie2015, allaire2022}.

\section{Results and discussion}\label{results-and-discussion}

\subsection{Bias}\label{bias}

\begin{figure}[!htb]

\centering{

\pandocbounded{\includegraphics[keepaspectratio]{access-ice_files/figure-pdf/fig-hindcast-extent-1.pdf}}

}

\caption{\label{fig-hindcast-extent}Row a: Pan-Antarctic daily mean
sea-ice extent for all hindcasts initialised on the first of each
calendar month for ACCESS-S1 (column 1; green) and ACCESS-S2 (column 2;
purple). Observed mean sea-ice extent in each corresponding hindcast
period is shown in black. Row b: Mean differences between the forecast
and the observed values. Circles represent the initial conditions at the
start of forecasts (i.e., the first of every month), and triangles
represent the mean values at the lead time corresponding to the maximum
lead time in S1 (between 213 and 216 days, depending on the month)}

\end{figure}%

Figure~\ref{fig-hindcast-extent} (Figure~\ref{fig-hindcast-extent-osi}
for OSI) shows mean sea-ice extent of the ACCESS-S1 and ACCESS-S2
hindcasts (row a) and their differences from mean sea-ice extent of
NOAA/NSIDC CDRV4 (row b). Mean extent at the first of every month is
indicated with circles for the initial conditions and with triangles for
the longest lead time possible for each model (between 274 and 277 days
for ACCESS-S2 and between 213 and 216 days for ACCESS-S1). At this long
lead time, information of the initial conditions is essentially lost and
the forecast reverts close to each model's preferred equilibrium state.

ACCESS-S2 initial conditions (circles in Fig.~\ref{fig-hindcast-extent}
column 2) show an overall negative bias, especially in the late
summer-early autumn, while ACCESS-S1 initial conditions (circles in
Fig.~\ref{fig-hindcast-extent} column 1) are very close to observations,
as expected from the assimilation of sea-ice observations to produce the
initial conditions of ACCESS-S1. Both systems' equilibrium states
(triangles) show negative biases of sea-ice extent, particularly in the
growth phase of late-autumn and winter months. This is due primarily to
the melt season being longer than in observations and with faster melt
between January and March and the growing seasons being shorter with
slower growth during March and April. This is then followed by faster
growth between May and July (Figure~\ref{fig-mean-growth} and
Figure~\ref{fig-mean-growth-osi}). Many sea-ice models exhibit this
systematic underestimation during the sea-ice minimum and early freezing
season \citep{massonnet2023}, which could indicate problems in the
representation of thermodynamics in the model \citep{zampieri2019}. It
is also not surprising that both forecasting systems converge to a
similar equilibrium state because they share the same model formulation.

The difference between the initial conditions (circles) and the model
equilibrium state (triangles) can be mostly attributed to the effect of
data assimilation, which in ACCESS-S2 is due solely to the coupling of
sea-ice with the atmosphere and the ocean. From April to September, in
ACCESS-S2 circles are closer to observations than the triangles are,
indicating that the information from the ocean and atmosphere data
assimilation is affecting sea ice and improving the initial conditions.
During these months, ACCESS-S1 can overestimate the sea-ice extent at
short lead time. For the rest of the year circles are overlaid with
triangles in ACCESS-S2, indicating that the ocean and atmosphere data
assimilation is not affecting sea ice and that this component of the
model is virtually free-running.

\begin{figure}[!htb]

\centering{

\pandocbounded{\includegraphics[keepaspectratio]{access-ice_files/figure-pdf/fig-mean-growth-1.pdf}}

}

\caption{\label{fig-mean-growth}Mean daily sea-ice extent growth
(\(10^6 km^2/day\)) in ACCESS-S1 (green) and ACCESS-S2 (purple)
hindcasts and observations (black), computed as the mean daily
differences in sea-ice extent between each date and the next for each
forecast month. Values are smoothed with a 11-day running mean.}

\end{figure}%

\begin{figure}[!htb]

\centering{

\pandocbounded{\includegraphics[keepaspectratio]{access-ice_files/figure-pdf/fig-bias-1-1.pdf}}

}

\caption{\label{fig-bias-1}expression(glue::glue(``Ensemble mean
difference between monthly sea-ice concentration of ACCESS-S2 ensemble
mean forecast at 0-month lead time (monthly mean values forecasted from
the forecast initialised at the first of the month) and observations
(CDR).''))}

\end{figure}%

\begin{figure}[!htb]

\centering{

\pandocbounded{\includegraphics[keepaspectratio]{access-ice_files/figure-pdf/fig-bias-2-1.pdf}}

}

\caption{\label{fig-bias-2}Same as Figure~\ref{fig-bias-1} but for
ACCESS-S1.}

\end{figure}%

To further understand the bias in ACCESS-S2, Figure~\ref{fig-bias-1}
(Figure~\ref{fig-bias-1-osi}) shows spatial patterns of the differences
of monthly mean sea-ice concentrations between NOAA/NSIDC CDRV4 and
ACCESS-S2 hindcasts at the shortest monthly lead time. From October to
May, the model underestimates sea-ice concentrations in most regions
except for the inner Weddell Sea in April and May, where sea-ice
concentrations saturate to 1 both in the observations and forecasts. In
winter, the differences are limited to a narrow band around the sea-ice
edge with slight positive biases in the African sector of East
Antarctica and negative biases around the Indian Ocean sector which
partially compensate, resulting in the near-zero extent bias seen in
those months (Figure~\ref{fig-hindcast-extent}).

ACCESS-S1 has a comparatively smaller overall bias
(Figure~\ref{fig-bias-2} and Figure~\ref{fig-bias-2-osi}). The largest
values are found between April and June, when the faster growth results
in large positive bias along the sea-ice edge, and in January, when the
faster melt leads to large negative bias in the Weddell and Amundsen
Seas.

\FloatBarrier

\subsection{RMSE}\label{rmse}

\begin{figure}[!htb]

\centering{

\pandocbounded{\includegraphics[keepaspectratio]{access-ice_files/figure-pdf/fig-extent-anom-1.pdf}}

}

\caption{\label{fig-extent-anom}Monthly mean sea-ice extent anomalies of
the observations (black) and forecasts from ACCESS-S1 (right column;
purple) and ACCESS-S2 (left column; green) at lead times of 0, 2, 4, and
6 months. The RMSE and correlation with their respective 95\% confidence
interval during the overlapping period of ACCESS-S1 and ACCESS-S2
hindcasts (1990--2013) are shown on the top left and bottom left of each
panel respectively}

\end{figure}%

Figure~\ref{fig-extent-anom} (Figure~\ref{fig-extent-anom-osi}) shows
monthly sea-ice extent anomalies forecasted at selected lead times.
Compared with ACCESS-S1, ACCESS-S2 anomaly forecasts are relatively poor
(large RMSE) even for the first month (lead time 0), whereas ACCESS-S1
forecasts stay relatively skilful even at a lead time of three months.
ACCESS-S2 shows much larger interannual variability than observations,
with dramatic lows between 1995 and 2007, and highs between 2007 and
2015.

Unexpectedly, for ACCESS-S2, RMSE improves with lead time, even though
the correlation degrades with lead time. This effect is seen in all
months except from July to September Figure~\ref{fig-extent-rmse-month}.
This is puzzling behaviour that goes contrary to what is usually seen in
prediction models. The explanation seems to be the mentioned increased
interannual variability. Figure~\ref{fig-extent-sd}
(Figure~\ref{fig-extent-sd-osi}) shows the interannual standard
deviation of monthly sea-ice extent of the forecasts as a function of
lead time compared with observations. ACCESS-S1 standard deviation lies
within the observed standard deviation regardless of lead time, while
ACCESS-S2 standard deviation is more than twice that of observations at
zero lead time and only approaches the observed value at nine month lead
time for most months.

\begin{figure}[!htb]

\centering{

\pandocbounded{\includegraphics[keepaspectratio]{access-ice_files/figure-pdf/fig-extent-sd-1.pdf}}

}

\caption{\label{fig-extent-sd}Interannual standard deviation with 95\%
confidence interval of monthly mean sea-ice extent forecasted for each
month. We standardise the standard deviation by that month's sea-ice
extent observation standard deviation. ACCESS-S1 and ACCESS-S2 at
different lead times. Each panel indicates the target month. Note the
reverse horizontal axis.}

\end{figure}%

ACCESS-S2 forecasts of sea-ice extent anomalies seem to align moderately
well with observations (leading to moderately high correlation) but
their magnitude is overestimated (leading to large errors). This could
be caused by ACCESS-S2 sea ice being much more sensitive to atmospheric
and oceanic forcing, perhaps due to lower thickness.

As an example, Figure~\ref{fig-forecast-example} shows sea-ice
concentration anomalies (top row) and sea-ice thickness and the
difference between the two models (bottom row) for 2 May 2008
initialised one day prior; being that close to initialisation date,
these are very approximately the initial conditions. ACCESS-S1 sea-ice
concentrations anomalies are very close to observations as expected from
the system assimilating these data. ACCESS-S2 sea-ice concentration
anomalies, which are not assimilated, are not as close, but the
large-scale pattern is aligned with observations. The system simulates
large positive anomalies in the Weddell and Ross Seas and slight
negative anomalies in the Amundsen and Bellingshausen Seas. The fact
that ACCESS-S2 can simulate this pattern without assimilating sea-ice
data suggests that atmospheric and oceanic forcing were the dominant
drivers. However, the magnitude of the sea-ice anomalies is too big. It
is plausible that this is due to the thinner ice simulated by ACCESS-S2
(bottom row).

\begin{figure}[!htb]

\centering{

\pandocbounded{\includegraphics[keepaspectratio]{access-ice_files/figure-pdf/fig-forecast-example-1.pdf}}

}

\caption{\label{fig-forecast-example}ACCESS-S1 and ACCESS-S2 hindcasts
for 2 May 2008 at one day lead time. Top row shows sea-ice concentration
anomalies forecasted by each system and the observations. Bottom row
shows forecasted sea-ice thickness and the difference between ACCESS-S1
and ACCESS-S2.}

\end{figure}%

\begin{figure}[!htb]

\centering{

\pandocbounded{\includegraphics[keepaspectratio]{access-ice_files/figure-pdf/fig-mean-thickness-1.pdf}}

}

\caption{\label{fig-mean-thickness}Mean and 95\% interval of monthly
mean sea-ice thickness for ACCESS-S1 and ACCESS-S2 at different lead
times. Each panel indicates the target month. Note the reverse
horizontal axis.}

\end{figure}%

Extending beyond the one case in Figure~\ref{fig-forecast-example},
Figure~\ref{fig-mean-thickness} shows monthly mean sea-ice thickness as
a function of lead time for ACCESS-S1 and ACCESS-S2. Supporting the idea
that thinner ice is what causes the increased extent variability in
ACCESS-S2, this system simulates thinner sea-ice compared to ACCESS-S1
overall at almost all lead times and in all months except for summer at
short lead times (Dec-Jan, 0-1 months; Feb-Mar, 0-2 months). However, in
both systems, forecasted sea-ice is thicker at shorter lead times and
then decreases, particularly in the summer months. If thinner ice were a
sufficient cause of increased variability, then we would expect
variability to increase with lead time in both forecasting systems.

The fact that ACCESS-S1 and ACCESS-S2 share the same model configuration
and that the increased variability is more extreme at short lead times
(Fig.~\ref{fig-extent-sd}) suggests that the data assimilation procedure
is partly responsible. It is possible that sea-ice in the ACCESS-S2
system is left in an unbalanced state after assimilating atmospheric and
oceanic data but not sea-ice data, leading to large responses that are
amplified by the thin ice in the initial states which then subside at
longer lead times when the model is balanced.

\begin{figure}[!htb]

\centering{

\pandocbounded{\includegraphics[keepaspectratio]{access-ice_files/figure-pdf/fig-rmse-1.pdf}}

}

\caption{\label{fig-rmse}Mean RMSE of sea-ice concentration anomalies as
a function of forecast lead time for all forecasts initialised on the
first of each month compared with a reference forecast of persistence of
anomalies (black) and climatology (gray). Only the first 120 days are
shown. In parentheses, the shortest time at which ACCESS-S1 and
ACCESS-S2 mean RMSE is not statistically different at the 99\%
confidence level.}

\end{figure}%

To assess ACCESS-S2 forecasts in more detail, we compute error measures
for all hindcasts started on the 1st of every month.
Figure~\ref{fig-rmse} (Figure~\ref{fig-rmse-osi}) shows the mean RMSE of
sea-ice concentration anomalies for ACCESS-S1 and ACCESS-S2 hindcasts
compared against persistence and climatological forecasts used as a
benchmark. Due to errors in the initial conditions, it is expected that
persistence forecasts would be better than the model forecasts at very
short lead times, but that the persistence forecast errors would grow
faster and may eventually surpass the model forecast errors. The black
line shows that the persistence forecast error indeed grows rapidly and
reaches its maximum in about 30 days for most months except for
February, when it grows much slower. The ACCESS-S1 forecast errors grow
slower than persistence forecast errors and remain lower after less than
10 days on average. The ACCESS-S2 forecast error starts high in all
months and is lower than the persistence forecast error after more than
15 days in most months except for forecast initialised in February, when
it takes 80 days.

At longer lead times, it is more appropriate to compare errors with the
climatological forecast error. The lead time at which ACCESS-S1 forecast
error is higher than the climatological forecast error varies between
more than 60 and less than 20 days depending on forecast initialisation
month with the minimum in June. ACCESS-S2 forecasts never have lower
error than climatology, on the other hand, except marginally in October
forecasts.

Figure~\ref{fig-lead-time-window}
(Figure~\ref{fig-lead-time-window-osi}) summarises the lead time window
in which each hindcast is better than both the persistence forecast and
the climatological forecast as a function of forecast month. ACCESS-S1
forecasts have a wider lead time window in the summer than the other
seasons and is not better than both benchmarks at forecasting June
sea-ice concentration anomalies. Forecasts initialised between May and
July are particularly poor, and June cannot be forecasted better than
the benchmarks. This is consistent with the mid-winter loss of
predictability observed by \citet{libera2022}, who attributed it to deep
warm water entraining into the mixed layer.

\begin{figure}[!htb]

\centering{

\pandocbounded{\includegraphics[keepaspectratio]{access-ice_files/figure-pdf/fig-lead-time-window-1.pdf}}

}

\caption{\label{fig-lead-time-window}Minimum lead time at which each
forecast's mean RMSE becomes larger than the lower bound of the 95\%
confidence interval of persistence forecast RMSE (black lines) and
maximum lead time at which each forecast's mean RMSE remains lower than
the lower bound of the 95\% confidence interval of climatological
forecast RMSE (gray lines). Green shading indicates the window where
forecasts outperform both persistence (lead times longer than black
line) and climatology (lead times shorter than gray line). Text labels
show the date corresponding to the maximum lead time at which each
forecast outperforms climatology.}

\end{figure}%

\begin{figure}[!htb]

\centering{

\pandocbounded{\includegraphics[keepaspectratio]{access-ice_files/figure-pdf/fig-rmse_lon-1-1.png}}

}

\caption{\label{fig-rmse_lon-1}MSE skill score of ACCESS-S1 forecasts
with climatological forecast as reference computed on 15 meridional
slices 24° wide as a function of lead time and longitude. Antarctica's
coastline is shown at the bottom of each panel for reference. Values
were smoothed with an 11-day running mean to improve readability.}

\end{figure}%

\begin{figure}[!htb]

\centering{

\pandocbounded{\includegraphics[keepaspectratio]{access-ice_files/figure-pdf/fig-rmse_lon-2-1.png}}

}

\caption{\label{fig-rmse_lon-2}Same as Figure~\ref{fig-rmse_lon-1} but
for ACCESS-S2.}

\end{figure}%

To analyse the spatial distribution of the model error, we computed the
RMSE of zonal mean sea-ice concentration anomalies on 15 slices of 24°
longitude span for each forecasting system. We control for some areas
being naturally easier to forecast than others by computing the RMSE
skill score with the climatological forecast RMSE as reference.

For ACCESS-S1 forecasts (Figure~\ref{fig-rmse_lon-1} and
Figure~\ref{fig-rmse_lon-1-osi}), skill tends to be lower off the coast
of Eastern Antarctica even at short lead times; for instance, the skill
score for forecasts initialised in May and June are negative between 0°
and 120°E even at almost zero lead time. This mirrors \citet{libera2022}
findings of a ``winter predictability barrier'', although they focus on
the Weddell Sea and here we show that the effect seems to be stronger
more to the east. In West Antarctica there is a hint of
easterly-propagating skill in forecasts initialised in February and
March. This is consistent with \citet{holland2013} findings that memory
of sea-ice anomalies are stored in ocean heat content anomalies that are
transported east by the Antarctic Circumpolar Current.

ACCESS-S2 forecasts (Figure~\ref{fig-rmse_lon-2} and
Figure~\ref{fig-rmse_lon-2-osi}) also have lower skill over East
Antarctica. From July to December even though the pan-Antarctic average
skill is negative at all lead times (Fig.~\ref{fig-lead-time-window}),
it is positive for up to a month in West Antarctica. Since oceanic and
atmospheric forcing is the only source of information, this suggests
that sea-ice in this region is particularly sensitive to oceanic and
atmospheric forcing and suggests a role of the Pacific-South American
mode and the Amundsen Sea Low to shape sea-ice concentration anomalies.
The fact that this is evident in the months in which El Niño--Southern
Oscillation teleconnections are more important for atmospheric
circulation also suggests the influence of tropical Pacific variability.
February and March are the only two months that can be forecasted with
marginally positive skill in large regions.

\begin{figure}[!htb]

\centering{

\pandocbounded{\includegraphics[keepaspectratio]{access-ice_files/figure-pdf/fig-rmse_lon-3-1.png}}

}

\caption{\label{fig-rmse_lon-3}Same as Figure~\ref{fig-rmse_lon-1} but
for the difference between ACCESS-S1 and ACCESS-S2.}

\end{figure}%

Finally, Figure~\ref{fig-rmse_lon-3} (Figure~\ref{fig-rmse_lon-3-osi})
shows the difference in skill between ACCESS-S1 and ACCESS-S2. Large
differences in skill indicate areas and months that are most affected by
the data assimilation present in ACCESS-S1. Between January and March,
which are the months in which ACCESS-S1 is the most skilful
(Figure~\ref{fig-lead-time-window}), most of the improvement compared
with ACCESS-S2 is present in the Ross and Weddell Sea. In April and May,
the improvement seems more homogeneous.

\begin{figure}[!htb]

\centering{

\pandocbounded{\includegraphics[keepaspectratio]{access-ice_files/figure-pdf/fig-initial-spread-1.pdf}}

}

\caption{\label{fig-initial-spread}Decomposition of forecast error
spread at 1, 5 and 30 days lead time for ACCESS-S1 and ACCESS-S2
hindcasts across initialization months. The left column shows the mean
standard deviation of RMSE errors across ensemble members, while the
right column shows the standard deviation of the ensemble mean RMSE
error and the spread of the persistence and climatology forecasts
errors.}

\end{figure}%

In Figure~\ref{fig-rmse} the mean error was shown.
Figure~\ref{fig-initial-spread} (Figure~\ref{fig-initial-spread-osi})
column 1 shows the mean standard deviation of errors among ensemble
members at various lead times. At one day lead time
(Fig.~\ref{fig-initial-spread} a.1) ACCESS-S2 has a slightly larger
spread than ACCESS-S1 due to the way that ensemble members are
generated. ACCESS-S1 ensemble members are generated by adding random
field perturbations to the atmosphere only, which then are transferred
to the other components via the coupled simulation \citep{hudson2017}.
With this scheme, ensemble members are all but guaranteed to be
underdispersed in the ocean and sea-ice components. The time-lag
ensemble used for ACCESS-S2 ensures greater spread. This difference is
gone after about just two days, and both systems have a comparable
spread in ensemble member error afterwards
(Fig.~\ref{fig-initial-spread} b1 and c1).

Figure~\ref{fig-initial-spread} column 2, on the other hand, shows the
standard deviation of ensemble mean error of each hindcast and the
persistence forecast. At one day lead time, ACCESS-S2 ensemble mean
error standard deviation is much larger than ACCESS-S1's, which in turn
is comparable to the persistence forecast error standard deviation. At
longer lead times, the spread of ACCESS-S1 and persistence forecast
standard deviation increases to eventually be comparable to ACCESS-S2
and the standard deviation in climatological forecast errors. ACCESS-S2
error standard deviation is fairly independent of lead time and similar
to the climatological forecast error standard deviation at all lead
times.

\FloatBarrier

\subsection{Conclusions}\label{conclusions}

Sea-ice forecasts from the ACCESS-S2 system show a significant low
extent bias, particularly during late summer and early autumn. This bias
is attributed to a faster and longer melt season between January and
March, and slower growth between March and April. This underestimation
during the minimum and early freezing season is a common issue in many
subseasonal-to-seasonal (S2S) systems, suggesting potential problems
either with the model's thermodynamic representation or with short wave
radiation forcing, as shown in other climate models
\citep{zampieri2019, roach2020}. Even though ACCESS-S2 shares the same
model components as ACCESS-S1, the latter does not suffer from this
bias, indicating that assimilating sea-ice concentrations successfully
corrects for the negative bias that exists in the free-running model.

Ensemble spread grows quickly even when perturbations are only
implemented in the atmosphere component (in ACCESS-S1), indicating that
sea ice is indeed responding quickly to atmospheric perturbations.
However, our analysis suggests that the atmosphere and ocean data
assimilation implemented in ACCESS-S2 is only effectively influencing
sea-ice initial conditions from June to October, while the rest of the
year, the sea-ice component runs virtually free, reverting to its biased
equilibrium state. \citet{zhou2022} had previously evaluated sea-ice
forecasts in ACCESS-S2 and also highlighted the poor performance of this
forecasting system attributed to the lack of good initial conditions.

Analysis of the error spread shows that ACCESS-S2 initial conditions
from December to May not only have large errors, but that the initial
error spread is very large compared with ACCESS-S1. This spread is not
due to the perturbation scheme, since the mean error variance for
individual forecasts is low and comparable with ACCESS-S1. Instead, it
is due to large variance of the mean error of individual forecasts,
which is comparable to the climatology spread. This is further evidence
that individual initial conditions are not being affected by the data
assimilation scheme.

Although ACCESS-S1 only assimilates sea-ice concentration, it is clear
that sea-ice thickness is also affected through the assimilation
process. ACCESS-S1 simulates significantly thicker ice than ACCESS-S2
and in both systems sea-ice is thicker at shorter lead times than at
longer lead times. Both the explicit data assimilation in ACCESS-S2 and
the effects of atmospheric and oceanic data assimilation in ACCESS-S1
might be nudging simulated sea ice to be thicker than the model
equilibrium state. We suggest that the thinner sea ice in ACCESS-S2
contributes to the large sea-ice extent variance, but other mechanisms,
such as unbalanced initial conditions might also be important.

Given that ACCESS-S2 sea-ice extent is not directly initialised by
sea-ice observations, comparing its forecasts with those of ACCESS-S1
allows us to estimate the time-scale over which initial conditions are
important. We find that initial conditions affect Antarctic sea-ice
forecasts in the order of a few months, but that effect is seasonally
dependent. January to April initial conditions improve forecasts for up
to three months. February initial conditions in particular are shown to
be crucial for determining sea-ice evolution at least up to May. Arctic
sea-ice forecasts also show greater sensitivity to initial conditions in
boreal summer, compared with boreal winter \citep{day2014, bunzel2016},
suggesting a similar mechanism might be playing a role.

Forecasts initialised in winter have very little skill and ACCESS-S1 and
ACCESS-S2 forecast errors are statistically indistinguishable after just
two weeks. This is consistent with \citet{libera2022}'s finding of a
``winter predictability barrier'' in the Weddell Sea, although they
describe the barrier as a sharp loss of predictability in July, and here
we find a gradual reduction in skill compared with climatology around
June. This difference might be due to our use of pan-Antarctic RMSE,
since our regional analysis indicates that the degraded skill is most
dramatic in the King Haakon Sea.

These findings have important implications for both operational
forecasting, model development and predictability studies. For
operational centers, our results suggest that efforts to improve sea-ice
data assimilation should prioritize the summer and autumn months when
initial conditions have the greatest impact on forecast skill.
Additionally, the substantial bias in ACCESS-S2 highlights the need for
improved model physics, particularly in the representation of sea-ice
thermodynamics and radiation processes. Crucially, our results suggest
dramatic seasonal variations in sea-ice predictability. Future studies
should therefore use initial conditions through the whole year rather
than focusing on a limited number of initialisation dates.

\section{References}\label{references}

\renewcommand{\bibsection}{}
\bibliography{references.bib}

\clearpage

\section{Appendix}\label{appendix}

\appendixfigures

The following are the same figures from the main paper but using the OSI
dataset instead of CDR.

\begin{figure}[!htb]

\centering{

\pandocbounded{\includegraphics[keepaspectratio]{access-ice_files/figure-pdf/fig-hindcast-extent-osi-1.pdf}}

}

\caption{\label{fig-hindcast-extent-osi}Row a: Pan-Antarctic daily mean
sea-ice extent for all hindcasts initialised on the first of each
calendar month for ACCESS-S1 (column 1; green) and ACCESS-S2 (column 2;
purple). Observed mean sea-ice extent in each corresponding hindcast
period is shown in black. Row b: Mean differences between the forecast
and the observed values. Circles represent the initial conditions at the
start of forecasts (i.e., the first of every month), and triangles
represent the mean values at the lead time corresponding to the maximum
lead time in S1 (between 213 and 216 days, depending on the month)}

\end{figure}%

\clearpage

\begin{figure}[!htb]

\centering{

\pandocbounded{\includegraphics[keepaspectratio]{access-ice_files/figure-pdf/fig-mean-growth-osi-1.pdf}}

}

\caption{\label{fig-mean-growth-osi}Mean daily sea-ice extent growth
(\(10^6 km^2/day\)) in ACCESS-S1 (green) and ACCESS-S2 (purple)
hindcasts and observations (black), computed as the mean daily
differences in sea-ice extent between each date and the next for each
forecast month. Values are smoothed with a 11-day running mean.}

\end{figure}%

\clearpage

\clearpage

\begin{figure}[!htb]

\centering{

\pandocbounded{\includegraphics[keepaspectratio]{access-ice_files/figure-pdf/fig-bias-1-osi-1.pdf}}

}

\caption{\label{fig-bias-1-osi}Ensemble mean difference between monthly
sea-ice concentration of ACCESS-S2 ensemble mean forecast at 0-month
lead time (monthly mean values forecasted from the forecast initialised
at the first of the month) and observations (OSI).}

\end{figure}%

\clearpage

\begin{figure}[!htb]

\centering{

\pandocbounded{\includegraphics[keepaspectratio]{access-ice_files/figure-pdf/fig-bias-2-osi-1.pdf}}

}

\caption{\label{fig-bias-2-osi}Same as Figure~\ref{fig-bias-1} but for
ACCESS-S1.}

\end{figure}%

\clearpage

\begin{figure}[!htb]

\centering{

\pandocbounded{\includegraphics[keepaspectratio]{access-ice_files/figure-pdf/fig-extent-anom-osi-1.pdf}}

}

\caption{\label{fig-extent-anom-osi}Monthly mean sea-ice extent
anomalies of the observations (black) and forecasts from ACCESS-S1
(right column; purple) and ACCESS-S2 (left column; green) at lead times
of 0, 2, 4, and 6 months. The RMSE and correlation with their respective
95\% confidence interval during the overlapping period of ACCESS-S1 and
ACCESS-S2 hindcasts (1990--2013) are shown on the top left and bottom
left of each panel respectively}

\end{figure}%

\clearpage

\begin{verbatim}
$data
$data[[1]]
     colour  x         y PANEL group flipped_aes linewidth linetype alpha
1   #9C59D1 -9 0.5984605     1     1       FALSE       0.5        1    NA
2   #9C59D1 -8 0.4772937     1     1       FALSE       0.5        1    NA
3   #9C59D1 -7 0.6183552     1     1       FALSE       0.5        1    NA
4   #9C59D1 -6 0.6039833     1     1       FALSE       0.5        1    NA
5   #9C59D1 -5 0.6250643     1     1       FALSE       0.5        1    NA
6   #9C59D1 -4 0.7092255     1     1       FALSE       0.5        1    NA
7   #9C59D1 -3 0.7552064     1     1       FALSE       0.5        1    NA
8   #9C59D1 -2 0.9081425     1     1       FALSE       0.5        1    NA
9   #9C59D1 -1 1.0889439     1     1       FALSE       0.5        1    NA
10  #9C59D1  0 1.1941767     1     1       FALSE       0.5        1    NA
11  #ACA40A -7 0.7565361     1     2       FALSE       0.5        1    NA
12  #ACA40A -6 0.5959845     1     2       FALSE       0.5        1    NA
13  #ACA40A -5 0.5955162     1     2       FALSE       0.5        1    NA
14  #ACA40A -4 0.4812004     1     2       FALSE       0.5        1    NA
15  #ACA40A -3 0.6866438     1     2       FALSE       0.5        1    NA
16  #ACA40A -2 0.4878597     1     2       FALSE       0.5        1    NA
17  #ACA40A -1 0.3958258     1     2       FALSE       0.5        1    NA
18  #ACA40A  0 0.2482179     1     2       FALSE       0.5        1    NA
19  #9C59D1 -9 0.4784391     2     1       FALSE       0.5        1    NA
20  #9C59D1 -8 0.4578146     2     1       FALSE       0.5        1    NA
21  #9C59D1 -7 0.4339634     2     1       FALSE       0.5        1    NA
22  #9C59D1 -6 0.4779146     2     1       FALSE       0.5        1    NA
23  #9C59D1 -5 0.5586558     2     1       FALSE       0.5        1    NA
24  #9C59D1 -4 0.5645042     2     1       FALSE       0.5        1    NA
25  #9C59D1 -3 0.5796230     2     1       FALSE       0.5        1    NA
26  #9C59D1 -2 0.8235551     2     1       FALSE       0.5        1    NA
27  #9C59D1 -1 0.8542623     2     1       FALSE       0.5        1    NA
28  #9C59D1  0 0.6718616     2     1       FALSE       0.5        1    NA
29  #ACA40A -7 0.5247870     2     2       FALSE       0.5        1    NA
30  #ACA40A -6 0.5405390     2     2       FALSE       0.5        1    NA
31  #ACA40A -5 0.4929244     2     2       FALSE       0.5        1    NA
32  #ACA40A -4 0.4194970     2     2       FALSE       0.5        1    NA
33  #ACA40A -3 0.4611708     2     2       FALSE       0.5        1    NA
34  #ACA40A -2 0.5108275     2     2       FALSE       0.5        1    NA
35  #ACA40A -1 0.2220359     2     2       FALSE       0.5        1    NA
36  #ACA40A  0 0.1245617     2     2       FALSE       0.5        1    NA
37  #9C59D1 -9 0.4207578     3     1       FALSE       0.5        1    NA
38  #9C59D1 -8 0.4177998     3     1       FALSE       0.5        1    NA
39  #9C59D1 -7 0.4165393     3     1       FALSE       0.5        1    NA
40  #9C59D1 -6 0.4958292     3     1       FALSE       0.5        1    NA
41  #9C59D1 -5 0.4837393     3     1       FALSE       0.5        1    NA
42  #9C59D1 -4 0.4255676     3     1       FALSE       0.5        1    NA
43  #9C59D1 -3 0.4514970     3     1       FALSE       0.5        1    NA
44  #9C59D1 -2 0.4883231     3     1       FALSE       0.5        1    NA
45  #9C59D1 -1 0.4978986     3     1       FALSE       0.5        1    NA
46  #9C59D1  0 0.4856632     3     1       FALSE       0.5        1    NA
47  #ACA40A -7 0.4558333     3     2       FALSE       0.5        1    NA
48  #ACA40A -6 0.4103865     3     2       FALSE       0.5        1    NA
49  #ACA40A -5 0.3550077     3     2       FALSE       0.5        1    NA
50  #ACA40A -4 0.3637343     3     2       FALSE       0.5        1    NA
51  #ACA40A -3 0.3655096     3     2       FALSE       0.5        1    NA
52  #ACA40A -2 0.2600277     3     2       FALSE       0.5        1    NA
53  #ACA40A -1 0.1856420     3     2       FALSE       0.5        1    NA
54  #ACA40A  0 0.1034259     3     2       FALSE       0.5        1    NA
55  #9C59D1 -9 0.4748831     4     1       FALSE       0.5        1    NA
56  #9C59D1 -8 0.5165270     4     1       FALSE       0.5        1    NA
57  #9C59D1 -7 0.5554201     4     1       FALSE       0.5        1    NA
58  #9C59D1 -6 0.5624090     4     1       FALSE       0.5        1    NA
59  #9C59D1 -5 0.4981750     4     1       FALSE       0.5        1    NA
60  #9C59D1 -4 0.5348628     4     1       FALSE       0.5        1    NA
61  #9C59D1 -3 0.6254053     4     1       FALSE       0.5        1    NA
62  #9C59D1 -2 0.6286052     4     1       FALSE       0.5        1    NA
63  #9C59D1 -1 0.6668942     4     1       FALSE       0.5        1    NA
64  #9C59D1  0 0.8824187     4     1       FALSE       0.5        1    NA
65  #ACA40A -7 0.4521898     4     2       FALSE       0.5        1    NA
66  #ACA40A -6 0.3974993     4     2       FALSE       0.5        1    NA
67  #ACA40A -5 0.4270625     4     2       FALSE       0.5        1    NA
68  #ACA40A -4 0.4026673     4     2       FALSE       0.5        1    NA
69  #ACA40A -3 0.3582458     4     2       FALSE       0.5        1    NA
70  #ACA40A -2 0.2900451     4     2       FALSE       0.5        1    NA
71  #ACA40A -1 0.2854501     4     2       FALSE       0.5        1    NA
72  #ACA40A  0 0.1821203     4     2       FALSE       0.5        1    NA
73  #9C59D1 -9 0.5579666     5     1       FALSE       0.5        1    NA
74  #9C59D1 -8 0.6127165     5     1       FALSE       0.5        1    NA
75  #9C59D1 -7 0.6029918     5     1       FALSE       0.5        1    NA
76  #9C59D1 -6 0.5841198     5     1       FALSE       0.5        1    NA
77  #9C59D1 -5 0.7928835     5     1       FALSE       0.5        1    NA
78  #9C59D1 -4 0.8556750     5     1       FALSE       0.5        1    NA
79  #9C59D1 -3 0.8069800     5     1       FALSE       0.5        1    NA
80  #9C59D1 -2 0.8437517     5     1       FALSE       0.5        1    NA
81  #9C59D1 -1 1.0196410     5     1       FALSE       0.5        1    NA
82  #9C59D1  0 1.1065527     5     1       FALSE       0.5        1    NA
83  #ACA40A -7 0.4807941     5     2       FALSE       0.5        1    NA
84  #ACA40A -6 0.5067542     5     2       FALSE       0.5        1    NA
85  #ACA40A -5 0.5393445     5     2       FALSE       0.5        1    NA
86  #ACA40A -4 0.4045959     5     2       FALSE       0.5        1    NA
87  #ACA40A -3 0.3845793     5     2       FALSE       0.5        1    NA
88  #ACA40A -2 0.3869069     5     2       FALSE       0.5        1    NA
89  #ACA40A -1 0.2904025     5     2       FALSE       0.5        1    NA
90  #ACA40A  0 0.3350960     5     2       FALSE       0.5        1    NA
91  #9C59D1 -9 0.6660617     6     1       FALSE       0.5        1    NA
92  #9C59D1 -8 0.6701446     6     1       FALSE       0.5        1    NA
93  #9C59D1 -7 0.7190907     6     1       FALSE       0.5        1    NA
94  #9C59D1 -6 0.9407559     6     1       FALSE       0.5        1    NA
95  #9C59D1 -5 1.0161669     6     1       FALSE       0.5        1    NA
96  #9C59D1 -4 0.8244582     6     1       FALSE       0.5        1    NA
97  #9C59D1 -3 0.8373873     6     1       FALSE       0.5        1    NA
98  #9C59D1 -2 0.9105125     6     1       FALSE       0.5        1    NA
99  #9C59D1 -1 0.8492269     6     1       FALSE       0.5        1    NA
100 #9C59D1  0 0.7665951     6     1       FALSE       0.5        1    NA
101 #ACA40A -7 0.5531898     6     2       FALSE       0.5        1    NA
102 #ACA40A -6 0.5026169     6     2       FALSE       0.5        1    NA
103 #ACA40A -5 0.4387233     6     2       FALSE       0.5        1    NA
104 #ACA40A -4 0.5029235     6     2       FALSE       0.5        1    NA
105 #ACA40A -3 0.4217662     6     2       FALSE       0.5        1    NA
106 #ACA40A -2 0.3496994     6     2       FALSE       0.5        1    NA
107 #ACA40A -1 0.3387964     6     2       FALSE       0.5        1    NA
108 #ACA40A  0 0.4231441     6     2       FALSE       0.5        1    NA
109 #9C59D1 -9 0.6450049     7     1       FALSE       0.5        1    NA
110 #9C59D1 -8 0.7003431     7     1       FALSE       0.5        1    NA
111 #9C59D1 -7 0.8720344     7     1       FALSE       0.5        1    NA
112 #9C59D1 -6 0.9102250     7     1       FALSE       0.5        1    NA
113 #9C59D1 -5 0.6764296     7     1       FALSE       0.5        1    NA
114 #9C59D1 -4 0.6273290     7     1       FALSE       0.5        1    NA
115 #9C59D1 -3 0.5766315     7     1       FALSE       0.5        1    NA
116 #9C59D1 -2 0.4778696     7     1       FALSE       0.5        1    NA
117 #9C59D1 -1 0.4421313     7     1       FALSE       0.5        1    NA
118 #9C59D1  0 0.3340629     7     1       FALSE       0.5        1    NA
119 #ACA40A -7 0.3926180     7     2       FALSE       0.5        1    NA
120 #ACA40A -6 0.4055218     7     2       FALSE       0.5        1    NA
121 #ACA40A -5 0.4615250     7     2       FALSE       0.5        1    NA
122 #ACA40A -4 0.4564888     7     2       FALSE       0.5        1    NA
123 #ACA40A -3 0.3854216     7     2       FALSE       0.5        1    NA
124 #ACA40A -2 0.3252267     7     2       FALSE       0.5        1    NA
125 #ACA40A -1 0.4481883     7     2       FALSE       0.5        1    NA
126 #ACA40A  0 0.3257066     7     2       FALSE       0.5        1    NA
127 #9C59D1 -9 0.5283287     8     1       FALSE       0.5        1    NA
128 #9C59D1 -8 0.6052768     8     1       FALSE       0.5        1    NA
129 #9C59D1 -7 0.6068318     8     1       FALSE       0.5        1    NA
130 #9C59D1 -6 0.4303770     8     1       FALSE       0.5        1    NA
131 #9C59D1 -5 0.4173341     8     1       FALSE       0.5        1    NA
132 #9C59D1 -4 0.3923987     8     1       FALSE       0.5        1    NA
133 #9C59D1 -3 0.3123127     8     1       FALSE       0.5        1    NA
134 #9C59D1 -2 0.3363830     8     1       FALSE       0.5        1    NA
135 #9C59D1 -1 0.2675447     8     1       FALSE       0.5        1    NA
136 #9C59D1  0 0.3365351     8     1       FALSE       0.5        1    NA
137 #ACA40A -7 0.3268014     8     2       FALSE       0.5        1    NA
138 #ACA40A -6 0.3875728     8     2       FALSE       0.5        1    NA
139 #ACA40A -5 0.4307579     8     2       FALSE       0.5        1    NA
140 #ACA40A -4 0.3423391     8     2       FALSE       0.5        1    NA
141 #ACA40A -3 0.3293139     8     2       FALSE       0.5        1    NA
142 #ACA40A -2 0.3821151     8     2       FALSE       0.5        1    NA
143 #ACA40A -1 0.3452069     8     2       FALSE       0.5        1    NA
144 #ACA40A  0 0.2554071     8     2       FALSE       0.5        1    NA
145 #9C59D1 -9 0.4812590     9     1       FALSE       0.5        1    NA
146 #9C59D1 -8 0.4743413     9     1       FALSE       0.5        1    NA
147 #9C59D1 -7 0.4297590     9     1       FALSE       0.5        1    NA
148 #9C59D1 -6 0.3962710     9     1       FALSE       0.5        1    NA
149 #9C59D1 -5 0.3600449     9     1       FALSE       0.5        1    NA
150 #9C59D1 -4 0.3618450     9     1       FALSE       0.5        1    NA
151 #9C59D1 -3 0.3679888     9     1       FALSE       0.5        1    NA
152 #9C59D1 -2 0.3721232     9     1       FALSE       0.5        1    NA
153 #9C59D1 -1 0.3502792     9     1       FALSE       0.5        1    NA
154 #9C59D1  0 0.3480131     9     1       FALSE       0.5        1    NA
155 #ACA40A -7 0.4420522     9     2       FALSE       0.5        1    NA
156 #ACA40A -6 0.4766696     9     2       FALSE       0.5        1    NA
157 #ACA40A -5 0.4193247     9     2       FALSE       0.5        1    NA
158 #ACA40A -4 0.4452902     9     2       FALSE       0.5        1    NA
159 #ACA40A -3 0.4194858     9     2       FALSE       0.5        1    NA
160 #ACA40A -2 0.3148970     9     2       FALSE       0.5        1    NA
161 #ACA40A -1 0.2736958     9     2       FALSE       0.5        1    NA
162 #ACA40A  0 0.1977477     9     2       FALSE       0.5        1    NA
163 #9C59D1 -9 0.4864481    10     1       FALSE       0.5        1    NA
164 #9C59D1 -8 0.4592050    10     1       FALSE       0.5        1    NA
165 #9C59D1 -7 0.4305820    10     1       FALSE       0.5        1    NA
166 #9C59D1 -6 0.3583702    10     1       FALSE       0.5        1    NA
167 #9C59D1 -5 0.3875961    10     1       FALSE       0.5        1    NA
168 #9C59D1 -4 0.3831390    10     1       FALSE       0.5        1    NA
169 #9C59D1 -3 0.4452131    10     1       FALSE       0.5        1    NA
170 #9C59D1 -2 0.3898225    10     1       FALSE       0.5        1    NA
171 #9C59D1 -1 0.4459450    10     1       FALSE       0.5        1    NA
172 #9C59D1  0 0.4557616    10     1       FALSE       0.5        1    NA
173 #ACA40A -7 0.5307289    10     2       FALSE       0.5        1    NA
174 #ACA40A -6 0.4979372    10     2       FALSE       0.5        1    NA
175 #ACA40A -5 0.5037229    10     2       FALSE       0.5        1    NA
176 #ACA40A -4 0.5176257    10     2       FALSE       0.5        1    NA
177 #ACA40A -3 0.3819866    10     2       FALSE       0.5        1    NA
178 #ACA40A -2 0.3222078    10     2       FALSE       0.5        1    NA
179 #ACA40A -1 0.2436012    10     2       FALSE       0.5        1    NA
180 #ACA40A  0 0.2279954    10     2       FALSE       0.5        1    NA
181 #9C59D1 -9 0.3978157    11     1       FALSE       0.5        1    NA
182 #9C59D1 -8 0.3974092    11     1       FALSE       0.5        1    NA
183 #9C59D1 -7 0.3097370    11     1       FALSE       0.5        1    NA
184 #9C59D1 -6 0.3419708    11     1       FALSE       0.5        1    NA
185 #9C59D1 -5 0.3765437    11     1       FALSE       0.5        1    NA
186 #9C59D1 -4 0.3672185    11     1       FALSE       0.5        1    NA
187 #9C59D1 -3 0.3915742    11     1       FALSE       0.5        1    NA
188 #9C59D1 -2 0.5181415    11     1       FALSE       0.5        1    NA
189 #9C59D1 -1 0.4613723    11     1       FALSE       0.5        1    NA
190 #9C59D1  0 0.6167969    11     1       FALSE       0.5        1    NA
191 #ACA40A -7 0.4989726    11     2       FALSE       0.5        1    NA
192 #ACA40A -6 0.4803765    11     2       FALSE       0.5        1    NA
193 #ACA40A -5 0.4394459    11     2       FALSE       0.5        1    NA
194 #ACA40A -4 0.3776623    11     2       FALSE       0.5        1    NA
195 #ACA40A -3 0.3166791    11     2       FALSE       0.5        1    NA
196 #ACA40A -2 0.2624609    11     2       FALSE       0.5        1    NA
197 #ACA40A -1 0.2813278    11     2       FALSE       0.5        1    NA
198 #ACA40A  0 0.1675816    11     2       FALSE       0.5        1    NA
199 #9C59D1 -9 0.3440246    12     1       FALSE       0.5        1    NA
200 #9C59D1 -8 0.3907403    12     1       FALSE       0.5        1    NA
201 #9C59D1 -7 0.3898249    12     1       FALSE       0.5        1    NA
202 #9C59D1 -6 0.3972901    12     1       FALSE       0.5        1    NA
203 #9C59D1 -5 0.4323504    12     1       FALSE       0.5        1    NA
204 #9C59D1 -4 0.4684049    12     1       FALSE       0.5        1    NA
205 #9C59D1 -3 0.6148242    12     1       FALSE       0.5        1    NA
206 #9C59D1 -2 0.5959746    12     1       FALSE       0.5        1    NA
207 #9C59D1 -1 0.6667716    12     1       FALSE       0.5        1    NA
208 #9C59D1  0 0.8636783    12     1       FALSE       0.5        1    NA
209 #ACA40A -7 0.4544102    12     2       FALSE       0.5        1    NA
210 #ACA40A -6 0.4791325    12     2       FALSE       0.5        1    NA
211 #ACA40A -5 0.4166218    12     2       FALSE       0.5        1    NA
212 #ACA40A -4 0.3897752    12     2       FALSE       0.5        1    NA
213 #ACA40A -3 0.4265074    12     2       FALSE       0.5        1    NA
214 #ACA40A -2 0.3779267    12     2       FALSE       0.5        1    NA
215 #ACA40A -1 0.3372106    12     2       FALSE       0.5        1    NA
216 #ACA40A  0 0.1711289    12     2       FALSE       0.5        1    NA


$layout
<ggproto object: Class Layout, gg>
    coord: <ggproto object: Class CoordCartesian, Coord, gg>
        aspect: function
        backtransform_range: function
        clip: on
        default: TRUE
        distance: function
        expand: TRUE
        is_free: function
        is_linear: function
        labels: function
        limits: list
        modify_scales: function
        range: function
        render_axis_h: function
        render_axis_v: function
        render_bg: function
        render_fg: function
        setup_data: function
        setup_layout: function
        setup_panel_guides: function
        setup_panel_params: function
        setup_params: function
        train_panel_guides: function
        transform: function
        super:  <ggproto object: Class CoordCartesian, Coord, gg>
    coord_params: list
    facet: <ggproto object: Class FacetWrap, Facet, gg>
        compute_layout: function
        draw_back: function
        draw_front: function
        draw_labels: function
        draw_panels: function
        finish_data: function
        init_scales: function
        map_data: function
        params: list
        setup_data: function
        setup_params: function
        shrink: TRUE
        train_scales: function
        vars: function
        super:  <ggproto object: Class FacetWrap, Facet, gg>
    facet_params: list
    finish_data: function
    get_scales: function
    layout: data.frame
    map_position: function
    panel_params: list
    panel_scales_x: list
    panel_scales_y: list
    render: function
    render_labels: function
    reset_scales: function
    resolve_label: function
    setup: function
    setup_panel_guides: function
    setup_panel_params: function
    train_position: function
    super:  <ggproto object: Class Layout, gg>

$plot
\end{verbatim}

\begin{verbatim}

attr(,"class")
[1] "ggplot_built"
\end{verbatim}

\begin{figure}[!htb]

\centering{

\pandocbounded{\includegraphics[keepaspectratio]{access-ice_files/figure-pdf/fig-extent-rmse-month-1.pdf}}

}

\caption{\label{fig-extent-rmse-month}RMSE of monthly mean sea-ice
extent anomalies as a function of lead time (months) for ACCESS-S1
(green) and ACCESS-S2 (purple). RMSE is computed over the overlapping
period of ACCESS-S1 and ACCESS-S2 hindcasts (1990--2013). Each panel
indicates the target month. Note the reverse horizontal axis.}

\end{figure}%

\clearpage

\begin{figure}[!htb]

\centering{

\pandocbounded{\includegraphics[keepaspectratio]{access-ice_files/figure-pdf/fig-extent-sd-osi-1.pdf}}

}

\caption{\label{fig-extent-sd-osi}Interannual standard deviation with
95\% confidence interval of monthly mean sea-ice extent forecasted for
each month. We standardise the standard deviation by that month's
sea-ice extent observation standard deviation. ACCESS-S1 and ACCESS-S2
at different lead times. Each panel indicates the target month. Note the
reverse horizontal axis.}

\end{figure}%

\clearpage

\begin{figure}[!htb]

\centering{

\pandocbounded{\includegraphics[keepaspectratio]{access-ice_files/figure-pdf/fig-rmse-osi-1.pdf}}

}

\caption{\label{fig-rmse-osi}Mean RMSE of sea-ice concentration
anomalies as a function of forecast lead time for all forecasts
initialised on the first of each month compared with a reference
forecast of persistence of anomalies (black) and climatology (gray).
Only the first 120 days are shown. In parentheses, the shortest time at
which ACCESS-S1 and ACCESS-S2 mean RMSE is not statistically different
at the 99\% confidence level.}

\end{figure}%

\clearpage

\begin{figure}[!htb]

\centering{

\pandocbounded{\includegraphics[keepaspectratio]{access-ice_files/figure-pdf/fig-lead-time-window-osi-1.pdf}}

}

\caption{\label{fig-lead-time-window-osi}Minimum lead time at which each
forecast's mean RMSE becomes larger than the lower bound of the 95\%
confidence interval of persistence forecast RMSE (black lines) and
maximum lead time at which each forecast's mean RMSE remains lower than
the lower bound of the 95\% confidence interval of climatological
forecast RMSE (gray lines). Green shading indicates the window where
forecasts outperform both persistence (lead times longer than black
line) and climatology (lead times shorter than gray line). Text labels
show the date corresponding to the maximum lead time at which each
forecast outperforms climatology.}

\end{figure}%

\clearpage

\clearpage

\begin{figure}[!htb]

\centering{

\pandocbounded{\includegraphics[keepaspectratio]{access-ice_files/figure-pdf/fig-rmse_lon-1-osi-1.png}}

}

\caption{\label{fig-rmse_lon-1-osi}MSE skill score of ACCESS-S1
forecasts with climatological forecast as reference computed on 15
meridional slices 24° wide as a function of lead time and longitude.
Antarctica's coastline is shown at the bottom of each panel for
reference. Values were smoothed with an 11-day running mean to improve
readability.}

\end{figure}%

\clearpage

\begin{figure}[!htb]

\centering{

\pandocbounded{\includegraphics[keepaspectratio]{access-ice_files/figure-pdf/fig-rmse_lon-2-osi-1.png}}

}

\caption{\label{fig-rmse_lon-2-osi}Same as Figure~\ref{fig-rmse_lon-1}
but for ACCESS-S2.}

\end{figure}%

\clearpage

\begin{figure}[!htb]

\centering{

\pandocbounded{\includegraphics[keepaspectratio]{access-ice_files/figure-pdf/fig-rmse_lon-3-osi-1.png}}

}

\caption{\label{fig-rmse_lon-3-osi}Same as Figure~\ref{fig-rmse_lon-1}
but for the difference between ACCESS-S1 and ACCESS-S2.}

\end{figure}%

\clearpage

\begin{figure}[!htb]

\centering{

\pandocbounded{\includegraphics[keepaspectratio]{access-ice_files/figure-pdf/fig-initial-spread-osi-1.pdf}}

}

\caption{\label{fig-initial-spread-osi}Decomposition of forecast error
spread at 1, 5 and 30 days lead time for ACCESS-S1 and ACCESS-S2
hindcasts across initialization months. The left column shows the mean
standard deviation of RMSE errors across ensemble members, while the
right column shows the standard deviation of the ensemble mean RMSE
error and the spread of the persistence and climatology forecasts
errors.}

\end{figure}%



\codedataavailability{\section{Code/Data
availability}\label{codedata-availability}

The underlying code for this study is available on GitHub:
https://github.com/eliocamp/access-s2\_ice-eval. Raw data of +S1 and +S2
hindcast are not available due to size. Derived datasets required to
reproduce the results, including extent timeseries and error measures,
are available in this Zenodo repository:
https://zenodo.org/records/17479538
\citep{campitelli2025}} %% use this section when having data sets and software code available



%%%%%%%%%%%%%%%%%%%%%%%%%%%%%%%%%%%%%%%%%%
%% optional

%%%%%%%%%%%%%%%%%%%%%%%%%%%%%%%%%%%%%%%%%%

%%%%%%%%%%%%%%%%%%%%%%%%%%%%%%%%%%%%%%%%%%
\authorcontribution{\section{Author
contributions}\label{author-contributions}

EC performed the data analysis and wrote the manuscript draft. AP, JA,
EL, MW and PR, performed interpretation of the results, and reviewed and
edited the draft. All authors read and approved the final
manuscript.} %% optional section

%%%%%%%%%%%%%%%%%%%%%%%%%%%%%%%%%%%%%%%%%%
\competinginterests{\section{Competing
interests}\label{competing-interests}

The authors declare no competing
interests.} %% this section is mandatory even if you declare that no competing interests are present

%%%%%%%%%%%%%%%%%%%%%%%%%%%%%%%%%%%%%%%%%%

%%%%%%%%%%%%%%%%%%%%%%%%%%%%%%%%%%%%%%%%%%
\begin{acknowledgements}
\section{Acknowledgements}\label{acknowledgements}

We thank the internal reviewers Bethan White and Xiaobing Zhou for their
comments and feedback. This work benefited from earlier unpublished work
by Laura Davies, Phil Reid, Andrew G. Marshall. This research was
undertaken with the assistance of resources from the National
Computational Infrastructure (NCI Australia), an NCRIS enabled
capability supported by the Australian Government. This work was
supported by ARC SRIEAS Grant SR200100005 Securing Antarctica's
Environmental Future.
\end{acknowledgements}

\bibliographystyle{copernicus}

\end{document}

% Options for packages loaded elsewhere
% Options for packages loaded elsewhere
\PassOptionsToPackage{unicode}{hyperref}
\PassOptionsToPackage{hyphens}{url}
\PassOptionsToPackage{dvipsnames,svgnames,x11names}{xcolor}
%
\documentclass[
  sn-mathphys-num,
  referee,lineno]{sn-jnl}


\usepackage{xcolor}
\usepackage{amsmath,amssymb}
\setcounter{secnumdepth}{-\maxdimen} % remove section numbering
\usepackage{iftex}
\ifPDFTeX
  \usepackage[T1]{fontenc}
  \usepackage[utf8]{inputenc}
  \usepackage{textcomp} % provide euro and other symbols
\else % if luatex or xetex
  \usepackage{unicode-math} % this also loads fontspec
  \defaultfontfeatures{Scale=MatchLowercase}
  \defaultfontfeatures[\rmfamily]{Ligatures=TeX,Scale=1}
\fi
\usepackage{lmodern}
\ifPDFTeX\else
  % xetex/luatex font selection
\fi
% Use upquote if available, for straight quotes in verbatim environments
\IfFileExists{upquote.sty}{\usepackage{upquote}}{}
\IfFileExists{microtype.sty}{% use microtype if available
  \usepackage[]{microtype}
  \UseMicrotypeSet[protrusion]{basicmath} % disable protrusion for tt fonts
}{}
\makeatletter
\@ifundefined{KOMAClassName}{% if non-KOMA class
  \IfFileExists{parskip.sty}{%
    \usepackage{parskip}
  }{% else
    \setlength{\parindent}{0pt}
    \setlength{\parskip}{6pt plus 2pt minus 1pt}}
}{% if KOMA class
  \KOMAoptions{parskip=half}}
\makeatother
% Make \paragraph and \subparagraph free-standing
\makeatletter
\ifx\paragraph\undefined\else
  \let\oldparagraph\paragraph
  \renewcommand{\paragraph}{
    \@ifstar
      \xxxParagraphStar
      \xxxParagraphNoStar
  }
  \newcommand{\xxxParagraphStar}[1]{\oldparagraph*{#1}\mbox{}}
  \newcommand{\xxxParagraphNoStar}[1]{\oldparagraph{#1}\mbox{}}
\fi
\ifx\subparagraph\undefined\else
  \let\oldsubparagraph\subparagraph
  \renewcommand{\subparagraph}{
    \@ifstar
      \xxxSubParagraphStar
      \xxxSubParagraphNoStar
  }
  \newcommand{\xxxSubParagraphStar}[1]{\oldsubparagraph*{#1}\mbox{}}
  \newcommand{\xxxSubParagraphNoStar}[1]{\oldsubparagraph{#1}\mbox{}}
\fi
\makeatother


\usepackage{longtable,booktabs,array}
\usepackage{calc} % for calculating minipage widths
% Correct order of tables after \paragraph or \subparagraph
\usepackage{etoolbox}
\makeatletter
\patchcmd\longtable{\par}{\if@noskipsec\mbox{}\fi\par}{}{}
\makeatother
% Allow footnotes in longtable head/foot
\IfFileExists{footnotehyper.sty}{\usepackage{footnotehyper}}{\usepackage{footnote}}
\makesavenoteenv{longtable}
\usepackage{graphicx}
\makeatletter
\newsavebox\pandoc@box
\newcommand*\pandocbounded[1]{% scales image to fit in text height/width
  \sbox\pandoc@box{#1}%
  \Gscale@div\@tempa{\textheight}{\dimexpr\ht\pandoc@box+\dp\pandoc@box\relax}%
  \Gscale@div\@tempb{\linewidth}{\wd\pandoc@box}%
  \ifdim\@tempb\p@<\@tempa\p@\let\@tempa\@tempb\fi% select the smaller of both
  \ifdim\@tempa\p@<\p@\scalebox{\@tempa}{\usebox\pandoc@box}%
  \else\usebox{\pandoc@box}%
  \fi%
}
% Set default figure placement to htbp
\def\fps@figure{htbp}
\makeatother





\setlength{\emergencystretch}{3em} % prevent overfull lines

\providecommand{\tightlist}{%
  \setlength{\itemsep}{0pt}\setlength{\parskip}{0pt}}





%%%% Standard Packages

\usepackage{graphicx}%
\usepackage{multirow}%
\usepackage{amsmath,amssymb,amsfonts}%
\usepackage{amsthm}%
\usepackage{mathrsfs}%
\usepackage[title]{appendix}%
\usepackage{xcolor}%
\usepackage{textcomp}%
\usepackage{manyfoot}%
\usepackage{booktabs}%
\usepackage{algorithm}%
\usepackage{algorithmicx}%
\usepackage{algpseudocode}%
\usepackage{listings}%

%%%%

\raggedbottom
\usepackage[section]{placeins}
\usepackage{xr}
\externaldocument{ref_main}
\makeatletter
\@ifpackageloaded{caption}{}{\usepackage{caption}}
\AtBeginDocument{%
\ifdefined\contentsname
  \renewcommand*\contentsname{Table of contents}
\else
  \newcommand\contentsname{Table of contents}
\fi
\ifdefined\listfigurename
  \renewcommand*\listfigurename{List of Figures}
\else
  \newcommand\listfigurename{List of Figures}
\fi
\ifdefined\listtablename
  \renewcommand*\listtablename{List of Tables}
\else
  \newcommand\listtablename{List of Tables}
\fi
\ifdefined\figurename
  \renewcommand*\figurename{Figure}
\else
  \newcommand\figurename{Figure}
\fi
\ifdefined\tablename
  \renewcommand*\tablename{Table}
\else
  \newcommand\tablename{Table}
\fi
}
\@ifpackageloaded{float}{}{\usepackage{float}}
\floatstyle{ruled}
\@ifundefined{c@chapter}{\newfloat{codelisting}{h}{lop}}{\newfloat{codelisting}{h}{lop}[chapter]}
\floatname{codelisting}{Listing}
\newcommand*\listoflistings{\listof{codelisting}{List of Listings}}
\makeatother
\makeatletter
\makeatother
\makeatletter
\@ifpackageloaded{caption}{}{\usepackage{caption}}
\@ifpackageloaded{subcaption}{}{\usepackage{subcaption}}
\makeatother
\usepackage{bookmark}
\IfFileExists{xurl.sty}{\usepackage{xurl}}{} % add URL line breaks if available
\urlstyle{same}
\hypersetup{
  pdftitle={The Importance of Initial Conditions in Seasonal Predictions of Antarctic Sea Ice},
  pdfauthor={Elio Campitelli; Ariaan Purich; Julie Arblaster; Eun-Pa Lim; Matthew C. Wheeler; Phillip Reid},
  pdfkeywords={sea ice, seasonal predictability, initial
conditions, forecasting},
  colorlinks=true,
  linkcolor={blue},
  filecolor={Maroon},
  citecolor={Blue},
  urlcolor={Blue},
  pdfcreator={LaTeX via pandoc}}


\title[The Importance of Initial Conditions in Seasonal Predictions of
Antarctic Sea Ice]{The Importance of Initial Conditions in Seasonal
Predictions of Antarctic Sea Ice}

% author setup
\author*[1,2]{\fnm{Elio} \sur{Campitelli}}\email{elio.campitelli@monash.edu}\author[1,2]{\fnm{Ariaan} \sur{Purich}}\author[1,2]{\fnm{Julie} \sur{Arblaster}}\author[3]{\fnm{Eun-Pa} \sur{Lim}}\author[3]{\fnm{Matthew C.} \sur{Wheeler}}\author[3]{\fnm{Phillip} \sur{Reid}}
% affil setup
\affil[1]{\orgdiv{School of Earth, Atmosphere and
Environment}, \orgname{Monash University, Kulin Nations, Clayton,
Victoria, Australia.}}
\affil[2]{\orgdiv{SARC Special Research Initiative for Securing
Antarctica's Environmental Future}, \orgname{Clayton, Kulin Nations,
Victoria, Australia}}
\affil[3]{\orgdiv{Research, Bureau of Meteorology, Melbourne,
Australia}}

% abstract 


% keywords
\keywords{sea ice,  seasonal predictability,  initial
conditions,  forecasting}

\begin{document}
\maketitle


\clearpage

\appendix  
\renewcommand{\thefigure}{A\arabic{figure}}
\setcounter{figure}{0}

\section{Supplementary figures}\label{supplementary-figures}

The following are the same figures from the main paper but using the OSI
dataset instead of CDR.

\begin{figure}[!htb]

\centering{

\pandocbounded{\includegraphics[keepaspectratio]{access-ice_files/figure-pdf/fig-hindcast-extent-osi-1.pdf}}

}

\caption{\label{fig-hindcast-extent-osi}Row a: Pan-Antarctic daily mean
sea-ice extent for all hindcasts initialised on the first of each
calendar month for ACCESS-S1 (column 1; green) and ACCESS-S2 (column 2;
purple). Observed mean sea-ice extent in each corresponding hindcast
period is shown in black. Row b: Mean differences between the forecast
and the observed values. Circles represent the initial conditions at the
start of forecasts (i.e., the first of every month), and triangles
represent the mean values at the lead time corresponding to the maximum
lead time in S1 (between 213 and 216 days, depending on the month)}

\end{figure}%

\begin{figure}[!htb]

\centering{

\pandocbounded{\includegraphics[keepaspectratio]{access-ice_files/figure-pdf/fig-mean-growth-osi-1.pdf}}

}

\caption{\label{fig-mean-growth-osi}Mean daily sea-ice extent growth
(\(10^6 km^2/day\)) in ACCESS-S1 (green) and ACCESS-S2 (purple)
hindcasts and observations (black), computed as the mean daily
differences in sea-ice extent between each date and the next for each
forecast month. Values are smoothed with a 11-day running mean.}

\end{figure}%

\begin{figure}[!htb]

\centering{

\pandocbounded{\includegraphics[keepaspectratio]{access-ice_files/figure-pdf/fig-bias-1-osi-1.pdf}}

}

\caption{\label{fig-bias-1-osi}Ensemble mean difference between monthly
sea-ice concentration of ACCESS-S2 ensemble mean forecast at 0-month
lead time (monthly mean values forecasted from the forecast initialised
at the first of the month) and observations (OSI).}

\end{figure}%

\begin{figure}[!htb]

\centering{

\pandocbounded{\includegraphics[keepaspectratio]{access-ice_files/figure-pdf/fig-bias-2-osi-1.pdf}}

}

\caption{\label{fig-bias-2-osi}Same as Figure \ref{fig-bias-1} but for
ACCESS-S1.}

\end{figure}%

\begin{figure}[!htb]

\centering{

\pandocbounded{\includegraphics[keepaspectratio]{access-ice_files/figure-pdf/fig-extent-anom-osi-1.pdf}}

}

\caption{\label{fig-extent-anom-osi}Monthly mean sea-ice extent
anomalies of the observations (black) and forecasts from ACCESS-S1
(right column; purple) and ACCESS-S2 (left column; green) at lead times
of 0, 2, 4, and 6 months. The RMSE and correlation during the
overlapping period of ACCESS-S1 and ACCESS-S2 hindcasts (1990--2013) are
shown on the top left and bottom left of each panel respectively.}

\end{figure}%

\begin{figure}[!htb]

\centering{

\pandocbounded{\includegraphics[keepaspectratio]{access-ice_files/figure-pdf/fig-extent-sd-osi-1.pdf}}

}

\caption{\label{fig-extent-sd-osi}Interannual standard deviation with
95\% confidence interval of monthly mean sea-ice extent forecasted for
each month divided by that month's sea-ice extent observation standard
deviation. ACCESS-S1 and ACCESS-S2 at different lead times. Each panel
indicates the target month. Note the reverse horizontal axis.}

\end{figure}%

\begin{figure}[!htb]

\centering{

\pandocbounded{\includegraphics[keepaspectratio]{access-ice_files/figure-pdf/fig-rmse-osi-1.pdf}}

}

\caption{\label{fig-rmse-osi}Mean RMSE of sea-ice concentration
anomalies as a function of forecast lead time for all forecasts
initialised on the first of each month compared with a reference
forecast of persistence of anomalies (black) and climatology (gray).
Only the first 120 days are shown. In parentheses, the shortest time at
which ACCESS-S1 and ACCESS-S2 mean RMSE is not statistically different
at the 99\% confidence level.}

\end{figure}%

\begin{figure}[!htb]

\centering{

\pandocbounded{\includegraphics[keepaspectratio]{access-ice_files/figure-pdf/fig-lead-time-window-osi-1.pdf}}

}

\caption{\label{fig-lead-time-window-osi}Minimum lead time at which each
forecast's mean RMSE becomes larger than the lower bound of the 95\%
confidence interval of persistence forecast RMSE (black lines) and
maximum lead time at which each forecast's mean RMSE remains lower than
the lower bound of the 95\% confidence interval of climatological
forecast RMSE (gray lines). Green shading indicates the window where
forecasts outperform both persistence (lead times longer than black
line) and climatology (lead times shorter than gray line). Text labels
show the date corresponding to the maximum lead time at which each
forecast outperforms climatology.}

\end{figure}%

\begin{figure}[!htb]

\centering{

\pandocbounded{\includegraphics[keepaspectratio]{access-ice_files/figure-pdf/fig-rmse_lon-1-osi-1.png}}

}

\caption{\label{fig-rmse_lon-1-osi}RMSE skill score of ACCESS-S1
forecasts with climatological forecast as reference computed on 15
meridional slices 24° wide as a function of lead time and longitude.
Antarctica's coastline is shown at the bottom of each panel for
reference.}

\end{figure}%

\begin{figure}[!htb]

\centering{

\pandocbounded{\includegraphics[keepaspectratio]{access-ice_files/figure-pdf/fig-rmse_lon-2-osi-1.png}}

}

\caption{\label{fig-rmse_lon-2-osi}Same as Figure \ref{fig-rmse_lon-1}
but for ACCESS-S2.}

\end{figure}%

\begin{figure}[!htb]

\centering{

\pandocbounded{\includegraphics[keepaspectratio]{access-ice_files/figure-pdf/fig-rmse_lon-3-osi-1.png}}

}

\caption{\label{fig-rmse_lon-3-osi}Same as Figure \ref{fig-rmse_lon-1}
but for the difference between ACCESS-S1 and ACCESS-S2.}

\end{figure}%

\begin{figure}[!htb]

\centering{

\pandocbounded{\includegraphics[keepaspectratio]{access-ice_files/figure-pdf/fig-initial-spread-osi-1.pdf}}

}

\caption{\label{fig-initial-spread-osi}Decomposition of forecast error
spread at 1, 5 and 30 days lead time for ACCESS-S1 and ACCESS-S2
hindcasts across initialization months. The left column shows the mean
standard deviation of RMSE errors across ensemble members, while the
right column shows the standard deviation of the ensemble mean RMSE
error and the spread of the persistence and climatology forecasts
errors.}

\end{figure}%


\bibliography{references.bib}



\end{document}
